\chapter*{Словарь терминов}             % Заголовок
\addcontentsline{toc}{chapter}{Словарь терминов}  % Добавляем его в оглавление

\textbf{Агент} --- пользователь или объект системы.

\textbf{Эффективность} --- оценочная характеристика РС.

\textbf{Объект} --- некоторая сущность предметной области РС.

\textbf{Пользователь} --- пользователь РС.

\textbf{Характеристика агента} --- единица метаданных агента. Например, жанр
объекта для кинематографической системы.

\textbf{Контент агента} --- структура данных, представляющая информацию об
агенте. Например, вектор.

\textbf{Рекомендательная система} --- программное обеспечение,
функциональность которых заключается в определении информации, которая
будет полезна пользователю.

\textbf{Модель рекомендательная система} --- математическая модель
рекомендательной системы, которая описывает способы представления информации
о пользователях и объектах, методы решения задач и взаимосвязи этой информации.

\textbf{Двумерная модель рекомендательная система} ---
наиболее распространенная модель, получившая свое название от определения
способа взаимосвязи информации о пользователе и объекте, которая
заключается в определении двумерной функции $\rho: U \times I \Rightarrow [0,1]$.

\textbf{Функция близости} --- функция $\rho: U \times I \Rightarrow [0,1]$,
значения которой определяют степень полезности, предпочтения и т.п. объекта
пользователю.

\textbf{Отношение близости пользователя и объекта} --- отношение, которое
формализует такие неформальные понятия как <<пользователю нравится объект>>,
<<объект полезен для пользователя>> и т.п. Факт выполнения отношения
устанавливается по значениям функции близости.

\textbf{Мера сходства пользователей} --- функция $\du: U \times U \Rightarrow
[0,1]$, определяющая степень близости пользователей по их характеристикам.

\textbf{Мера сходства объектов} --- функция $\di: I \times I \Rightarrow
[0,1]$, определяющая степень близости объектов по их характеристикам.

\textbf{Отношение близости объектов} --- отношение, которое определяет
близость агентов системы, определяемую по значениям меры сходства.

\textbf{Активный пользователь} --- пользователь системы, обозначающийся
символом $u_a$, для которого в данный момент решается задача.

\textbf{Задача topN} --- задача, целью которой является формирование
подмножества объектов, между которыми и активным пользователем выполняется
отношение близости.

\textbf{Задача прогнозирования} --- задача, обозначающаяся символом $p$ и
заключающаяся в определении неизвестного значения $\rho(u_a, i)$.

\textbf{Оценка качества} --- функция $\mathcal{E}_t$ оценки качества решения
задачи $t \in \{topn, p\}$.

