\chapter*{Список сокращений и условных обозначений}             % Заголовок
\addcontentsline{toc}{chapter}{Список сокращений и условных обозначений}  % Добавляем его в оглавление
\noindent
\addtocounter{table}{-1}% Нужно откатить на единицу счетчик номеров таблиц, так как следующая таблица сделана для удобства представления информации по ГОСТ
%\begin{longtabu} to \dimexpr \textwidth-5\tabcolsep {r X}
	\begin{longtable}{p{3cm}p{13cm}}
	$\rho$ & оценка близости пользователя и объекта\\
	$\Delta_{\R}$ & пороговое значение оценки близости $\rho$, по которому
	устанавливается выполнение отношения близости\\
		$\Delta_{u}$ & пороговое значение меры сходства $\du$, по которому
	устанавливается выполнение отношения близости $\ru$\\
		$\Delta_{i}$ & пороговое значение меры сходства $\di$, по которому
	устанавливается выполнение отношения близости $\rt$\\
	$I$ & множество объектов \\
	$U$ & множество пользователей \\
	$i$ & объект \\
	$i_0$ & объект, принадлежащий обучающему множеству \\
	$i_{\bot}$ & объект, принадлежащий тестовому множеству. Также
	прогнозируемый объект. \\
	$u$ & пользователь \\
	$t$ & задача\\
	$\overline{P}^a_{\bot}$ & результирующее множество \\
	$task$ & множество задач\\
	$u_a$ & активный пользователь \\
	$p$ & задача прогнозирования \\
	$topN$ & задача $topN$ \\
	$\mathbb{N}$ & множество натуральных чисел\\
	$\varepsilon_p$ & пороговое значение малой величины разницы между оценками
	близости или оценкой близости и значением прогнозной функции
	\\
	$\R$ & отношение близости пользователя и объекта \\
	$P$ & множество исходных данных, которое содержит все известные значения оценки близости \\
	$P_0$ & обучающее множество \\
	$P_{\bot}$ & тестовое множество \\
	$\mathcal{M}_{\rho}$ & матрица значений оценки близости\\
	$\bot$ & неизвестное значение\\
	$w_U$ & весовая функция, сопоставляющая характеристике пользователя вес\\
	$w_I$ & весовая функция, сопоставляющая характеристике объекта вес\\
	$\mathcal{S})_I$ & шкала значений весов характеристик объектов\\
	$\mathcal{S})_U$ & шкала значений весов характеристик пользователей\\
	$X$ & множество всех возможных характеристик пользователей \\
	$Y$ & множество всех возможных характеристик объектов \\
	ООМ & Объектно-ориентированныя модель \\
	СОМ & Субъектно-ориентированныя модель \\
	$I_{topN}$ & множество, содержащее объекты, по алгоритмическому
	выводу РС между которыми и активным пользователем выполняется отношение
	близости. Данное множество формируется в процессе решения задачи $topN$ \\
	$\rh$ & прогнозная функция \\
	$s$ & функция, сопоставляющая объекту 1, если между ним и активным
	пользователем выполняется отношение близости, 0 --- иначе. Применяется при вычислении
	целевой оценки точности\\
	$\overline{s}$ & функция, сопоставляющая объекту 1, если между ним и активным
	пользователем выполняется отношение близости по правилам вывода ООМ,
	0 --- иначе. Применяется при вычислении объектно-ориентированной оценки точности\\
	$|S|$ & мощность некоторого множества $S$\\
	$\eat$ & обобщенная целевая оценка точности \\
	$\eit$ & обобщенная объектно-ориентированная оценка точности \\
	$I^a_0$ & $\{i_0: (i | \rho(u_a, i_0)) \in P_0$ \\
	$\nit$ & кластер соседей-объектов, составляемый при решении задачи $topN$ в
	ООМ\\
	$\nip$ & кластер соседей-объектов, составляемый при решении задачи $p$ в
	ООМ\\
	$\nut$ & кластер соседей-объектов, составляемый при решении задачи $topN$ в
	СОМ\\
	$\nup$ & кластер соседей-объектов, составляемый при решении задачи $p$ в
	СОМ\\
	$\Pi$ & правило алгоритмического вывода значения прогнозной функции\\
	$\Pi_O$ & правило алгоритмического вывода значения прогнозной функции ООМ\\
	$\Pi_C$ & правило алгоритмического вывода значения прогнозной функции СОМ\\
	$\Pi_f$ & правило алгоритмического вывода значения прогнозной функции
	нечеткой модели\\
	$f_p$ & функция, выражающая зависимость между значением прогнозной функции
	и значениями оценок близости пользователей, вошедших в кластер $\nup$ю\\
	$s(i)$ & функция, принимающая значение, равное 1, если $u_a \R
	i, i \in I_{topN}, 0$ --- иначе. Применяется для расчета $\eat$.\\
	$\overline{s}(i)$ & функция, принимающая значение, равное 1,
	если $\exists i_{\bot}: i_{\bot} \rt i, i \in I_{topN}, 0$ --- иначе.
	Применяется для расчета $\eit$.\\
\end{longtable}
