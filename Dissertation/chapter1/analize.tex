\section{Анализ коллаборативной фильтрации и оценок}
Рассматриваемые модели характеризуются отсутствием единого математического
аппарата, в существующих исследованиях зачастую разнятся обозначения:
к примеру, в одних работах оценка сходства обозначается как
sim\cite{item-based},
в других --- $similarity$\cite{amazon-item2item}, в третьих ---
$w$\cite{emperical-cf}. Также разнятся названия техник как было показано в
разделе <<Объектно-ориентированные техники>>.
В основном, исследования
предлагают новые идеи без проведения математического анализа предлагаемых
или существующих моделей.

Существующие коллаборативных техники основаны на эвристических утверждениях,
выполнение которых необходимо для того, чтобы предлагаемые решения давали
эффективный результат. Однако анализ возможности выполнения или не выполнения,
или поиск необходимых условий выполнения этих утверждений не проводился.
Однако, если
существует возможность того, что они не выполняются, то существует возможность
потенциальной неэффективности предлагаемого
решения.

Исследователи используют неформальное понятие сходства, которое
применяется как отношение, например, когда говорится о сходстве соседей с
активным пользователем.
Однако свойства
этого отношения не анализируются. Далее, во второй главе, показано, что важным свойством, от которого 
зависят рассматриваемые модели, является свойство транзитивности этого отношения.  

В качестве мер сходства используются различные функции. Выбор той или иной функции так же является эвристическим.
К примеру, количественным показателем сходства элементов являются метрические функции\cite{fdca}. Существуют исследования, 
в которых проводится сравнительный анализ эффективности решения в зависимости от выбора функции оценки сходства\cite{content_rs_soc_sys}, 
что демонстрирует зависимость эффективности от этого выбора и важность этого этапа при реализации КРС.

В следующей главе будет дан формальный анализ существующих моделей во введенной терминологии.

Независимо от схемы решения и задачи, решения КРС производится расчет
меры сходства,
поэтому будем считать данную операцию элементарной при вычислении асимптотической сложности. 

При решении задачи $topN$ или прогнозирования необходимо построить матрицу оценок сходства $\mathcal{M}$. Асимптотическая
сложность ее равна $O(|I|^2)$ \cite{amazon-item2item}, так как необходимо произвести столько раз расчет оценки сходства
 $\di(i_1,i_2)$.

При решении задачи top-$N$ или прогнозирования производится построение множества соседей. В худшем случае эта асимптотическая 
сложность этой операции равна $O(|U|)$\cite{amazon-item2item}.

Реальные системы, к примеру, Amazon работают с огромным числом пользователей (свыше 29 миллионов) и объектов (свыше миллиона)\cite{amazon-size,netflix-size}, 
поэтому, в общем случае, асимптотическая сложность решений велика, и проблема масштабируемости имеет место быть\cite{scalable1,scalable2,scalable3,scalable4}.

Для каждого класса задач (прогнозирование, top-$N$) существует некоторый набор оценок сходства.

\begin{assert}
	\label{no-standard-eval}
	Выбор определенной оценки является проблемой отсутствия стандарта; оценки,
принадлежащие
различным классам не коррелируют между собой, поэтому невозможно сравнить две
рекомендательные
системы, оцененные различными методами \cite{herloker-eval}.
\end{assert}
