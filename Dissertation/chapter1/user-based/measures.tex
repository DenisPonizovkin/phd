
\subsubsection{Функции вычисления прогнозной оценки}
Приведем примеры распространенных функций $f_p$, которые используются для
вычисления значения прогнозной функции по значениям $\rho(u, i), u \in \nup$.
\begin{itemize}
  \item Среднее  \cite{surveyCf}:
  \begin{equation}
	\label{middle-pred}
    \frac{1}{|\nup|} \cdot \sum \limits_{u \in \nup} \rho(u,i_{\bot}).
  \end{equation}
  \item Среднее взвешенное \cite{surveyCf}:
  \begin{equation}
	  \label{weighted-pred}
    \overline{\rho^a} +
    \frac{ \sum \limits_{u \in \nup} \du(u_a,u) \cdot (\rho(u,i_{\bot}) -
	  \overline{\rho^u}) }
{ \sum \limits_{u \in \nup} | \du(u_a,u) | }
  \end{equation}
  $\overline{\rho^a}$ --- среднее значение близости активного пользователя,
		$\overline{\rho^u}$ --- среднее значение близости пользователя $u$.
Вычитание средней оценки близости призвано устранить эффект,
который накладывается от характера пользователя:
		лояльные пользователи ставят оценки близости не ниже определенной,
строгие пользователи --- наоборот \cite{norm, rs-handbook}.

Приведенные способы вычисления прогнозной оценки являются распространенными, но не единственными.
К примеру, компания Ringo \cite{ringo} не использовала в своей системе веса,
а среднее значение меры сходства соседей. Однако
метод среднего взвешенного является не только простым и распространенным,
но и согласуется с
теорией общественного выбора, ее обоснованиями человеческого поведения, ---
первичными принципами коллаборативной фильтрации  \cite{cfrs,bellcore}.
\end{itemize}

%%%%%%%%%%%%%%%%%%%%%%%%%%%%%%%%%%%%%%%%%%%%%%%%%%%%%%%%%
\subsubsection{Меры сходства}
Приведем примеры распространенных функций, которые используются в качестве меры
сходства.
\begin{itemize}
\item Коэффициент корреляции Пирсона\label{pearson}:
\begin{equation}
	\du(u,v)\frac{\sum \limits_{i \in I_{\bigcap}}(\rho(u,i) - \overline{\rho}^u) \cdot
	(\rho(v,i) - \overline{\rho}^v)}
                {\sqrt{\sum \limits_{i \in I^u}(\rho(u,i) - \overline{\rho})^u \cdot
                    \sum \limits_{i \in I^v}(\rho(v,i) - \overline{\rho}^v)^2}}
\end{equation}
\begin{equation*}
	I^a = \{i: (i, \rho(a,i)) \in P^a_0\}, a \in \{u,v\},P^a_0 = \{\rho(a, i_0)\}
\end{equation*}
\begin{equation*}
	I_{\bigcap} = I^u \bigcap I^v
\end{equation*}
\item Косинус угла между контентами.
	Для применения этой меры сходства необходимо представлять
контенты как элементы векторного пространства, в котором координата
		соответствует объекту и равна нулю, если пользователь не оценивал
		данный объект.
\begin{equation}
cos(\angle(u,v) = \frac{\sum \limits_{ i \in I_{\bigcap} } \rho(u,i) \cdot \rho(v,i)}{
\sqrt{ \sum \limits_{i \in  I^u } (\rho(u,i))^2 } \cdot \sqrt{ \sum \limits_{ i \in I^v} (\rho(u,i))^2 }},
\end{equation}
\end{itemize}
