
\subsubsection{Примеры решения задач}
Рассмотрим примеры решения задач в СОМ.
\paragraph{Данные}
\begin{itemize}
\item $N=2$ --- для задачи $topN$ требуется определить 2 топовых объекта;
\item $i_{\bot} = 7$ --- для задачи прогнозирования необходимо спрогнозировать оценку на объект 7;
\item $\mathcal{S} = \{0,1;0,2;0,3;0,4;0,5\}$ --- оценки принадлежат относительной шкале от 1 до 5. Если оценка равна 0; то пользователь не
ставил оценку на данный объект.
\item $|I| = 10$
\item Контенты пользователей --- вектор оценок, где $i-ая$ координата соответствует $i$-ому объекту.
  \begin{itemize}
    \item $a =   (0,5; 0,2; 0,4; 0,3; 0,5; 0,0; 0,0; 0,0; 0,0; 0,0)$
    \item $u_1 = (0,5; 0,2; 0,5; 0,0; 0,4; 0,2; 0,5; 0,0; 0,0; 0,0)$
    \item $u_2 = (0,4; 0,3; 0,5; 0,0; 0,4; 0,0; 0,0; 0,4; 0,5; 0,2)$
    \item $u_3 = (0,2; 0,5; 0,2; 0,4; 0,3; 0,0; 0,0; 0,5; 0,4; 0,0)$
    \item $u_4 = (0,0; 0,0; 0,0; 0,0; 0,0; 0,2; 0,0; 0,3; 0,5; 0,5)$
    \item $u_5 = (0,4; 0,2; 0,0; 0,0; 0,4; 0,2; 0,5; 0,5; 0,0; 0,0)$
  \end{itemize}
\item $\Delta_{\R} = 0,49$
\item В качестве меры сходства будем использовать коэффициент корреляции
	Пирсона (\ref{pearson}).
\item
  \begin{equation}
    \du(u_1,u_2) > 0.8 \Rightarrow u_1 \ru u_2
  \end{equation}
\end{itemize}
\paragraph{Задача $topN$}
\begin{enumerate}
\item Составим множество соседей мощности 2. Для этого рассчитаем меру сходства между активным пользователем и другими:
  \begin{itemize}
  \item $\du(u_a,u_1) = 0,94 \Rightarrow u_1 \in \nut$
  \item $\du(u_a,u_2) = 0,57$
  \item $\du(u_a,u_3) = -0,85$
  \item $\du(u_a,u_4) = \varnothing$
  \item $\du(u_a,u_5) = 0,81 \Rightarrow u_5 \in \nut$
  \end{itemize}
\item $s = \sum \limits_{u \in \nut} u = (0,9; 0,4; 0,5; 0,0; 0,8; 0,4; 1,0; 0,5; 0,0; 0,0)$
\item Зануляем значения координат, для которых $\rho(u_a, i) \in P_0: s =
	(0,0; 0,0; 0,0; 0,0; 0,0;0,4; 1,0; 0,5; 0,0; 0,0)$
\item $I_{topN} = \{ 7, 8 \}$
\end{enumerate}
\paragraph{Задача прогнозирования}
При решении предыдущей задачи было составлено множество соседей.
Определим прогноз по оценкам этих соседей:
\begin{equation}
f_p(u_a,7) =  0,38 + \frac{ 0,94 \cdot (0,5 - 0,38) + 0,81 \cdot (0,5 - 0,36)}{
	0,94 + 0,81 } = 0,55
\end{equation}
