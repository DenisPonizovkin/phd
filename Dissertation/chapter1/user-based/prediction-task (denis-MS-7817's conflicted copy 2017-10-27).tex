
\subsubsection{Задача прогнозирования}
Подобно задаче $topN$, для решения задачи прогнозирования \cite{cf-expert,
coscial-rec-survey, cfrs, toward,rs-handbook, user-item-cf,rs-cf}
строится кластер соседей, центром которого является активный пользователь
$u_a$, однако в кластер входят не только те пользователи, между которыми
и активным выполняется отношение близости, но и такие, которые оценили
прогнозируемый объект $i_{\bot}$:
$\nup = \{u: u_a \ru u \wedge \rho(u,i_{\bot}) \in P_0\}$. Такое дополнительное
условие накладывается для того, чтобы по известным $\rho(u, i_{\bot}), u \in \nup$
определить неизвестную $\rho(u,i_{\bot})$.
Следуя утверждению СОМ (\ref{assertSRS1}), значение $|\rho(u_a,i_{\bot}) -
\rho(u,i_{\bot})| \le
\varepsilon_p, u \in \nup$.
Для того, чтобы рассчитать прогнозную оценку,
вычисляется значение некоторой прогнозной функции $f_p$
от оценок, поставленных прогнозируемому объекту соседями:
\begin{equation}
  \rh(u_a,i_{\bot}) = f_p\{ \rho(u,i_{\bot}): u \in \nup \})
\end{equation}

\begin{figure}[h]
	\label{p-solve-srs}
	\begin{algorithm}
		\caption{Стандартный алгоритм решения задачи прогнозирования неизвестного значения
		$\rho(u_a,i_{\bot})$ СОМ}
		\begin{algorithmic}[1]
			\State $k = 0$
			\State $\nup \gets \varnothing$
			\For {$u \gets 1 \to |U|$}
			\If{ $(u_a \ru u) \wedge (\exists \rho(u, i_{\bot})) \wedge (k < M)$
			} \Comment{$M$ --- максимальное число пользователей в кластере}
			\State $\nup \gets \nup \bigcup \{ u \}$
			\EndIf
			\EndFor
			\State $\rho(u_a,i_{\bot}) \gets f_p\{ \rho(u,i_{\bot}) \in \nup \})$
		\end{algorithmic}
	\end{algorithm}
\end{figure}
