%\newcommand {\Rho}{\mathrm{P}}
\section{Основные термины и обозначения}
\subsection{Исходные данные}
Если описать РС неформальным языком, то можно сказать, что РС ---
это такие системы, с помощью которых {\it пользователь} может найти:
\begin{itemize}
	\item список {\it объектов}, которые будут ему нравится, либо
		будут каким-то образом полезны, предпочтительны и т.д.;
\item степень симпатии, полезности, предпочтительности и т.д. конкретного
	{\it объекта}.
\end{itemize}
{\it Объект РС} --- некоторая сущность предметной области РС. Это может быть,
к примеру, фильм (как для системы IMDB \cite{imdb}), музыкальный исполнитель
(как для системы LastFm \cite{lastfm}), товар (как для системы
Amazon \cite{amazon}) и т.п.

Объединим неформальные термины симпатия, полезность, предпочтение и т.д
в один термин --- {\it близость}, который используется в таких
областях компьютерных наук, как, например, кластерный анализ \cite{fdca} или
информационный поиск \cite{ir1,ir2,ir3,ir4}. Численный показатель близости
будем называть
{\it оценкой близости}, им является значение {\it функции
близости} $\rho: U \times I \rightarrow [0,1]$, где
$U = \{1,...,m\}$ --- множество идентификаторов пользователей, являющиеся
натуральными числами.
$I = \{1,...,n\}$ --- множество идентификаторов объектов, являющихся
натуральными числами.
Тип идентификатора
 вид могут быть разными для каждой системы, в зависимости от
выбора разработчиков. Для простоты будем считать, что идентификатор
--- натуральное число. Будем ассоциировать пользователя или объекта с его
идентификатором для простоты изложения и уменьшения числа нотаций.
Значения функции близости могут
быть заданы самими пользователями за время работы с системой в качестве
или быть вычислены РС. Область значений функции близости может быть отличной
от отрезка $[0,1]$ (что зависит от шкалы оценок пользователей в реальной
системе), например, $[-10,10]$  \cite{jester}.
Однако значение функции близости при работе с иной областью
значений всегда можно нормировать \cite{norm}
и привести к отрезку $[0,1]$. Область значений $[0,1]$ была принята для
удобства изложения выводов диссертационного исследования.
Будем считать, что чем {\it меньше} значение оценки близости, тем объект ближе, а, значит,
он более предпочтителен, симпатичен, полезен и т.д.
Будем говорить, что между пользователем $u$ и
объектом $i$ выполняется отношение близости $\R$, если
$\rho(u, i) \ge \varepsilon_{\R}$, где $\varepsilon_{\R}$ --- некоторая
задаваемая величина.
Будем называть таких пользователей и объектов {\it близкими}.

В общем виде, любая РС работает с информацией о значениях
функции $\rho(u, i)$, так как эта информация является
ключевой информацией. Поэтому исходными данными любой РС
является множество значений оценки близости.
Будем представлять исходные данные в виде множества
$P = \{(u, i, \rho(u, i)): \rho(u, i) \ne \bot\}$,
где символ $\bot$ означает неизвестное значение.
Часто в исследованиях РС исходные данные представляются в виде
матрицы оценок близости \cite{sparse1,sparse2,sparse3}:
\begin{center}
$\mathcal{M}_{\rho} =
\begin{pmatrix}
	\rho(1,1)& ... & \rho(1,n)  \\
	...      & \bot & ...  \\
	\rho(m,1)& ... & \rho(m, n)  \\
\end{pmatrix}$.\\
\end{center}
Как правило, матрица
$\mathcal{M}_{\rho}$ является разреженной \cite{sparse1, sparse2, sparse3},
то есть большинство
значений расстояний в ней неизвестно.

Формально целевую функциональность РС можно определить как
экстраполяцию функции $\rho$ \cite{toward}. Для реализации этой функциональности
применяются различные подходы, о которых пойдет речь далее. Однако
любые подходы анализируют информацию о метаданных пользователей или объектов.
Единицу метаданных будем называть {\it характеристикой}.
К примеру, характеристиками пользователя могут быть такие метаданные, как
пол и возраст, а характеристиками объекта --- наименование
кинематографического или музыкального жанра. Характеристики объектов
варьируются, их вид зависит от прикладной области РС, от реализации РС
и т.п, однако вид характеристик объектов не влияет на методики решений.
Поэтому в существующих исследованиях виду характеристик объектов не уделяется внимание.
Обозначим символом $X$ множество всех возможных характеристик пользователей.
Символом $Y$ --- множество всех возможных характеристик объектов.

Значением характеристики пользователя
является значение весовой функции $w_U: U \times X \rightarrow \mathcal{S}_U$,
объекта --- $w_I: I \times Y \rightarrow \mathcal{S}_I$. Значения весов могут задаваться
пользователями, экспертами, алгоритмически и т.д. $\mathcal{S}_U,
\mathcal{S}_I$ --- шкалы,
которые могут принадлежать одному из трех типов \cite{social_osipov}:
\begin{itemize}
\item бинарной;
\item порядковой;
\item относительной.
\end{itemize}

Для рассматриваемых в исследовании АКМ характеристиками
пользователей являются идентификаторы объектов, то есть
$X = I$, характеристиками объектов --- идентификаторы пользователей,
то есть $Y = U$, а $w_U(u, i) = w_I(i, u) = \rho(u, i)$
\cite{rs-in-compsciense}.

Структуру данных, которая содержит и представляет информацию о характеристиках агента, назовем
{\it контентом}. Обозначим $c_X(u)$ контент пользователя $u$, $c_Y(i)$ ---
контент пользователя $i$. В исследованиях применяются различные структуры для
представления данных об агентах. Контентом может быть вектор \cite{vsm1},
выборка  \cite{rs-handbook} и т.п.
Стандартно в исследованиях, посвященных коллаборативной фильтрации
контенты именуются профилями.
В описываемом
исследовании используется термин контент в более широком понимании
профиля, где множества характеристик могут быть не только $U$
или $I$.

\subsection{Задачи}
Разреженность матрицы $\mathcal{M}_{\rho}$ \cite{sparse1, sparse2, sparse3} является причиной
существования проблемы поиска нужной информации.
Иначе, если бы матрица была полностью заполнена, то любая система (не
обязательно РС) могла
бы выполнять простые запросы к базе данных для определения
множества объектов, близких пользователю, или для определения
значения оценки близости для заданного пользователя и объекта.

Необходимость заполнения матрицы $\mathcal{M}_{\rho}$ послужила толчком для
возникновения и развития РС как инструмента, способного снизить степень
разреженности для каждого пользователя (называемого в таком случае активным
и обозначаемого символом $u_a$) путем решения следующих двух задач:
\begin{enumerate}
	\item прогнозирования $pred$. По этой задаче требуется
		спрогнозировать неизвестное значение $\rho(u_a, i_{\bot})$
		путем алгоритмического вывода значения
		прогнозной функции $\rh(u_a, i_{\bot}): U \times I \rightarrow [0,1]$.

		%При этом требуется, чтобы прогнозирование было
		%составлено точно, то есть $|\rh(u_a, i_{\bot}) - \rho(u_a, i_{\bot})| \le
		%\varepsilon_p$;
		%Прогнозирование является одним из методов
		%экстраполяции, которая, в свою очередь является целевой
		%функциональностью РС;

\item $topN$. По этой задаче требуется сформировать подмножества объектов
	$I_{topN} = \{i: (u_a \R i) \wedge \rho(u_a, i) = \bot\}
		\wedge |I_{topN}| = N$.
		%Так как неизвестно, выполняется
		%ли отношение $u_a \R i$ в силу того, что $\rho(u_a, i) = \bot$,
		%то выполнение отношения $u_a \R i$ определяется по значению
		%прогнозной функции.
\end{enumerate}
%%%%%%%%%%%%%%%%%%%%%%%%%%%%%%%%%%%%%%%%%%%%%%%%%%%%%%%%%%%%%%%%%%
\subsection{Оценка качества решения}
Чтобы определить эффективность по критерию качества или стабильности
проводится тестирование. Для этого исходное множество данных $P$
разбивается на обучающее и тестовое множества, которые обозначим символами
$P_0$ и $P_{\bot}$ соответственно.
Если $(u, i, \rho(u, i)) \in P_0$, то будем обозначать такие объекты $i_0$.
Если $(u, i, \rho(u, i)) \in P_{\bot}$, то будем обозначать такие объекты $i_{\bot}$.
Как правило обучающие
множество составляют 80\% от исходных, тестовые --- 20\%  \cite{8020-1,
8020-2}. Множество $P_0$ на этапе
тестирования становится множеством исходных данных.

После получения результирующего множества
$\overline{P}_{\bot} = \{(u, i, \rh(u_a, i))\}$ в ходе решения задачи
по данным обучающего множества, проводится
сравнение результирующего множества с тестовым.
Сравнение производится
с помощью функций, называемых {\it оценками качества}. Для каждой задачи существует
своя группа оценок, в которую входит некоторое число функций.
Например,
некоторые из оценок задачи прогнозирования  \cite{rs-handbook,rs-eval-shani} --- это $MAE$, $NMAE$, $RMSE$,
из оценок задачи $topN$  \cite{rs-handbook,herloker-eval,eval-precision} ---
точность $P$, точность $P@L$ по списку длины $L$, средняя точность
$P@L$, $NDCG$. Чем меньше значение оценки качества, тем качество решения выше.
Будем говорить, что решение задачи $t$ качественно,
если $\mathcal{E}_{t}(\overline{P}_{\bot}, P_{\bot}) \le \varepsilon_{t},$ где
$t \in \{topN, pred\}$, $\mathcal{E}_{t}$ --- оценка качества
решения задачи $t$, $\varepsilon_{t}$ --- некоторая фиксированная
величина.
%%%%%%%%%%%%%%%%%%%%%%%%%%%%%%%%%%%%%%%%%%%%%%%%%%%%%%%%%%%%%%%%%%
\subsection{Определение модели рекомендательной системы}
Существуют различные подходы, применяемые для формирования решения задач:
кластерный анализ  \cite{cluster1,cluster2,cluster3,cluster4,cluster5,cluster6, cluster7, cluster8},
статистический  \cite{stat1, stat2, stat3, stat4, stat5, stat6, stat7, stat8, stat10, stat11, stat12},
машинное обучение  \cite{learning4, learning1, learning2, learning3, learning5, learning6, learning7, learning8, learning9, learning10, learning11, learning12},
контентный анализ  \cite{content8, content1, content3, content4, content6,
content7, content9, content10, content11, content12}, коллаборативная
фильтрация
 \cite{cfrs,
item-based, 2d, empirical-cf, coscial-rec-survey, surveyCf, Marlin04collaborativefiltering}.

Используются различные модели РС:
векторная  \cite{empirical-cf},
 \cite{e-commerce,vsm1}, временные \cite{temporal-model},
вероятностные \cite{bayesian-model},
двумерная, трехмерная \cite{2d} и т.д. В исследовании рассматриваются
коллаборативные модели, которые используют технику коллаборативной фильтрации.
Эти модели являются одними из
наиболее изученных  \cite{most-researched, rs-in-compsciense},
популярных  \cite{most-popular, rs-in-compsciense} и успешных техник
\cite{most-success}.

Модель РС --- это четверка:
\begin{equation}
	\label{model}
	(c_X; c_Y; \Pi; \mathcal{E}_{t}),
\end{equation} где
$\Pi$ --- правило алгоритмического вывода значения прогнозной функции $\rh$.
Введенное определение включает в себя определение модели РС,
данное в работе \cite{2d},
так как определение неизвестной оценки близости покрывается заданием
правила вывода.
Введенное определение модели (\ref{model})
содержит в себе способ представления данных пользователя и объекта $c_X, c_Y$,
что является важным, так как от него зависит то, какие меры близости могут быть
использованы для работы с контентами. К примеру, если $c_Y$ задается в
виде вектора, то можно использовать функцию косинуса.
К тому же правило вывода и вид контентов могут быть связаны друг с другом,
и задание только правила вывода полностью не определит модель.
Также в описание модели
входит и способ оценки ее эффективности по критерию качества, так как
эффективность является неотъемлимым свойством модели.

\subsection{Коллаборативная фильтрация}
В данном исследовании речь идет об анамнестических методах коллаборативной фильтрации
\cite{cfrs,
item-based, 2d, empirical-cf, coscial-rec-survey, surveyCf, Marlin04collaborativefiltering}
как о правиле вывода $\Pi$ модели.
Модели РС, которые используют данные методы в качестве правила вывода $\Pi$,
будем обозначать АКМ.
Характеристики пользователей в данных системах жестко определены:
$X = I$, а значение $w_U(u, i) = \rho(u, i) \in P$.
Оценка близости $\rho$, являясь значением
характеристики, принадлежат либо бинарной,
либо порядковой, либо относительной шкале.
Бинарная шкала оценок может быть
использована, к примеру, в интернет-магазинах \cite{amazon-item2item} и определять
факт приобретения товара: если значение
оценки равно 0 (не 1, так как определяется близость и чем значение меньше, тем
товар более предпочтителен для пользователя), то пользователь приобрел
соответствующий товар.

Ключевым понятием АКМ, на базе которого строятся правила вывода $\Pi$,
является понятие {\it близости} пользователей или объектов системы.
Близость --- не формализуемый термин, он ``... не свободен от смыслового
многообразия, а его синонимами являются понятия <<подобие>>, <<близость>>,
<<связанность>>, <<ассоциативность>>`` \cite{fdca}. АКМ проводит анализ
степеней близости пользователей или объектов. Степень близости является значением
функции, именуемой
{\it мерой близости}(similarity measure)
 \cite{toward}.

Область определения меры близости может быть разной,
в зависимости от того, между элементами которого множества
рассчитывается близость: между пользователями, либо между объектами.
Область определения задает тип фильтрации: по множеству пользователей или по
множеству объектов.
Мера близости, используемая для определения значений близости пользователей:
\begin{equation}
	\du : U \times U \rightarrow [0,1]
\end{equation}

Мера близости, используемая для определения значения близости объектов:
\begin{equation}
	\di : I \times I \rightarrow [0,1]
\end{equation}

Область значений мер близости функции принадлежит отрезку $[0,1]$.
Область значений меры близости в практических реализациях РС, работающей с пользователями,
может быть иной, но его значение всегда можно нормировать \cite{norm}
и привести к отрезку [0,1]. Область значений [0,1] была принята для
удобства изложения выводов диссертационного исследования.

Аксиомами АКМ, на которых основаны методы решений задач, являются эвристические утверждения \cite{item-based,
topn2, person-rec-item-based, heur1, heur2, heur3},
поэтому	АКМ также
называют эвристическими \cite{heur3,heur1,heur2}.

АКМ делятся на два типа, в зависимости от того, какое утверждение
используется, от чего зависит и область определения оценки близости. То есть
АКМ делятся на следующих два типа: объектно-ориентированная модель (далее ООМ), когда
производится фильтрация множества объектов при применении $\di$, и
субъектно-ориентированная модель (далее СОМ), когда используется функция $\du$
Далее более подробно опишем каждый из типов.
