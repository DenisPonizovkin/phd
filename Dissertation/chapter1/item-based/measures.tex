
\subsubsection{Функции вычисления прогнозной оценки}
Приведем
пример распространенной функции прогнозирования, применяемой в ООМ
 \cite{item-based}:
%cite
\begin{equation}\label{fpredict-ors}
	\frac{1}{|\nip|} \frac{\sum \limits_{i \in \nip} \di(i,i_{\bot}) \cdot \rho(u_a,i)}{
\sum \limits_{i \in \nip} \di(i,i_{\bot})
}
\end{equation}
%===================================================================
%--------------------------------------------
%===================================================================
\subsubsection{Используемые функции в качестве меры сходства}
\begin{itemize}
% ToDo: information retireval bib
\item \textbf{Косинус.} \newline
Наиболее популярная мера сходства, заимствованная из области
информационного поиска \cite{ir1,ir2,ir3}:
\begin{equation} \label{sim-cos}
	cos(\angle(c_Y(i),c_Y(j)) = \frac{\sum \limits_{k=1}^{|Y|} y^i_k \cdot y^j_k}{
		\sqrt{ \sum \limits_{k=1}^{|Y|} (y^i_k)^2 } \cdot
		\sqrt{ \sum \limits_{k=1}^{|Y|} (y^j_k)^2 }},
\end{equation}

\item {\bf Условная вероятность.} \newline
Данная мера сходства была разработана Кариписом \cite{topn1} для такой
предметной области РС, в которой оценки близости принадлежат бинарной
шкале. Оценки близости обладают семантикой условной вероятности
события (к примеру, приобретения товара),
которое зависит от истории поведения пользователя:
$Pr( \rho(u_a,i) = 1 | \rho(u_a,j) = 1 )$
\begin{equation}
Pr(i,j) = \frac{\tf(i \wedge j)}{\tf(i) \cdot \tf(j)^{\alpha}}
\end{equation}
где:
\begin{itemize}

\item $\tf(i)$ --- весовой коэффициент, использование которого заимствовано из
области информационного поиска, этот коэффициент определяет частоту
(от term frequency), с какой объект оценивается пользователями, то есть
$\tf(i) = |\{\rho(u,i): \rho(u,i) \in P_0\}|$,
$\tf(i ^ j) = |U^i \bigcup U^j|$,
$U^i = \{u: \rho(u_a,i) \in P_0\}$,
$U^j = \{u: \rho(u,j) \in P_0\}$.

\item $\alpha$ --- коэффициент, который служит для учета популярных объектов,
имеющих, к примеру, тенденцию быть приобретенными большинством пользователей
(и иметь, таким образом, большое значение частоты).

\end{itemize}

\item {\bf Коэффициент корреляции Пирсона.} Мера сходства,
заимствованная из статистики. Данная мера сходства использовалась
компаниями GroupLens \cite{grouplens} и Bellcore \cite{bellcore}.
\begin{equation}
 \frac{\sum \limits_{y}(y^1 - \overline{y^1}) \cdot (y^2 -
                                           \overline{y^2})}
                {\sqrt{\sum \limits_{y}(y^1 - \overline{y^1})^2 \cdot
        \sum \limits_{y}(y^2 - \overline{y^2})^2}},
\end{equation}

$\overline{y^k}$ -- среднее значение характеристики $k$-ого объекта.
\end{itemize}
