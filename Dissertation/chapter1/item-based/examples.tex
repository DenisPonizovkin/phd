
\subsubsection{Примеры решения задач}
Рассмотрим примеры решения задач помощью ООМ.
\paragraph{Данные}
\begin{itemize}
\item
	Пусть контенты объектов некоторой системы представляют
		собой вектора в трехмерном пространстве характеристик $Y =
		\{y_1,y_2,y_3\}$, значения характеристик
	--- действительные числа.
	Пусть в системе существуют объекты со следующими контентами:
  \begin{enumerate}
  \item $c_Y(1) = (0,2; 0,8; 0)$;
  \item $c_Y(2) = (0; 1; 0)$;
  \item $c_Y(3) = (0,2; 1; 0)$;
  \item $c_Y(4) = (0; 0,2; 0)$;
  \item $c_Y(5) = (1; 0; 0)$;
  \item $c_Y(6) = (0,8; 0; 0)$;
  \item $c_Y(7) = (0; 0; 1)$;
  \item $c_Y(8) = (0; 0; 0,8)$;
  \item $c_Y(9) = (0; 0; 1)$;
  \end{enumerate}
\item Матрица мер сходства:\\
$\mathbb{M} = $
$
\begin{pmatrix}
0    & 0.97 & 0.99   & 0.97 & 0.24 & 0.19 & 0 & 0    & 0    \\
0.97 & 0    & 0.98   & 1    &   0  &  0   & 0 & 0    & 0    \\
0.99 & 0.98 & 0      & 0.97 & 0.58 & 0.19 & 0 & 0    & 0    \\
0.97 & 1    & 0.97   & 0    & 0.37 & 0    & 0 & 0    & 0    \\
0.24 & 0    & 0.58   & 0.37 & 0    & 1    & 0 & 0    & 0    \\
0.19 & 0    & 0.19   & 0    & 1    & 0    & 0 & 0    & 0    \\
0    & 0    & 0      & 0    & 0    & 0    & 0 & 0.98 & 1    \\
0    & 0    & 0      & 0    & 0    & 0    & 0.98 & 0 & 0.98    \\
0    & 0    & 0      & 0    & 0    & 0    & 1    & 0.98 & 0    \\
\end{pmatrix}
$
\item $c_X(u_a) = (1;1,0,0,0,0,1,0,0)$
\item $N = 2$ --- для задачи $topN$ нужно определить
	два объекта, между которыми и активным пользователем выполняется отношение
		близости;
\end{itemize}

\paragraph{Решение задачи $topN$}
\begin{enumerate}
\item В матрице $\mathcal{M}$ в каждом столбце оставляем $N$
	наибольших элементов:
$\mathcal{M'} = $
$
\begin{pmatrix}
0    & 0,97 & 0,99   & 0,97 &      & 0,19 & 0 & 0    & 0    \\
0,97 & 0    & 0,98   & 1    &   0  &  0   & 0 & 0    & 0    \\
0,99 & 0,98 & 0      &      & 0,58 & 0    & 0 & 0    & 0    \\
0    & 1    & 0      & 0    &      & 0    & 0 & 0    & 0    \\
0    & 0    & 0      & 0    & 0    & 1    & 0 & 0    & 0    \\
0    & 0    & 0      & 0    & 1    & 0    & 0 & 0    & 0    \\
0    & 0    & 0      & 0    & 0    & 0    & 0 & 0,98 & 1    \\
0    & 0    & 0      & 0    & 0    & 0    & 0,98 & 0 & 0,98    \\
0    & 0    & 0      & 0    & 0    & 0    & 1    & 0,98 & 0    \\
\end{pmatrix}
$
\item Перемножим {\it транспонированный} контент $c_X(u_a)^T$
	пользователя и матрицу $\mathcal{M'}$
		$m = (u_a)^T \times \mathcal{M'} = s = (0,97; 0,97, 1.97, 1.97, 0, 0,19, 0, 0,98, 1)$
\item Из вектора $m$ выбираем два наибольших значения. Им
	соответствуют объекты 3 и 4. Поэтому
		$I_{topN} = \{3, 4\}$.
\end{enumerate}

\paragraph{Задача прогнозирования}
Решим задачу прогнозирования для объекта 4 и активного пользователя со
следующими данными $P^a_0 = \{(2 | 1), (3, 0,1)\}$.
\begin{enumerate}
	\item $i_{\bot} = 4$ --- необходимо спрогнозировать оценку 4-ого объекта;
	\item Будем пользоваться функцией прогнозирования, описанной формулой
		(\ref{pred-f}).
	\item Составим множество соседей прогнозируемого объекта 4, для чего
		воспользуемся матрицей $\mathcal{M}$, $\nip = \{2, 3\}$
\item $\rho(u_a, 4) = \frac{1 \cdot 0.97 + 0,9 \cdot 1}{|0.97 + 1|} = 0,95$
\end{enumerate}
