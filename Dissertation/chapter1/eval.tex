%eval-sim-metrics ToDo: раписать целевую
\section{Описание функций оценок качества решений задач} \label{def-eval}

\subsection{Оценка решения задачи прогнозирования}
Данные оценки определяют, насколько {\it аккуратно} был произведен прогноз путем
расчета погрешности между спрогнозированной оценкой близости и реальной,
принадлежащей тестовому множеству.

Приведем распространенные оценки эффективности этого класса:
\begin{itemize}
\item
  \begin{equation}
	  MAE = \frac{1}{|P^a_{\bot}|} \sum \limits_{(u_a, i_{\bot}, \rho(u_a,
	  i_{\bot})) \in P^a_{\bot}} |\rho(u_a, i_{\bot})
	  - \rh(u_a, i_{\bot})|,
  \end{equation}
		где $P^a_{\bot} = \{\rho(u_a, i_{\bot})\}$ --- тестовое множество
		активного пользователя;
\item
  \begin{equation}
	  \label{nmae}
	  NMAE = \frac{1}{|P_{\bot}|} \cdot (\rho_{\max} - \rho_{\min})
	  \sum \limits_{i \in I_{\bot}} |\du(u_a, i^m_{\bot}) - \rho(u_a, i^m_{\bot})|,
  \end{equation}
$\rho_{\max}, \rho_{\min}$ --- максимальная и минимальная оценки
шкалы соответственно.

Значения $NMAE$ труднее интерпретировать по отношению к масштабу шкалы,
но они сопоставимы для шкал любого масштаба.
\item
  \begin{equation}
	  RMSE = \sqrt{\frac{1}{|P^a_{\bot}|} \sum \limits_{(u_a, i_{\bot}, \rho(u_a, i_{\bot})) \in P^a_{\bot}}
	  (\rh(u_a, i_{\bot}) - \rho(u_a, i_{\bot}))^2}
  \end{equation}
\end{itemize}

%Решение задачи прогнозирования можно рассматривать как формирование системой
%прототипа пользователя $\overline{a}$,
%в контент которого входят спрогнозированные значения.
%Тогда цель системы заключается в том, чтобы сформировать контент пользователя
%$\overline{a}$ так, что
%$a \ru \overline{a}$, то есть $\forall i_{\bot}$ $|\rho(u_a,i_{\bot}) -
%\overline{\rho}(\overline{a},i_{\bot})| \le \epsilon_0$,
%$\overline{\rho}(u_a,i_{\bot})$ --- спрогнозированная оценка.
%Но тогда и оценки эффективности будут иметь значение меньше либо равное
%$\epsilon_0$. Поэтому будем говорить, что решение эффективно, если
%значение оценки эффективности меньше либо равно $\epsilon_0$.

\subsection{Оценка решения задачи $topN$}
% ToDo добавить еще P@ и всякие NDCG
Делая запрос на решение задачи $topN$, активный пользователь $u_a$
ставит перед системой цель
определить множество $I_{topN}$ таких объектов $i$, что $u_a \R i$. Для того,
чтобы определить, насколько точно по отношению к цели было составлено множество
$I_{topN}$ применяются функции, именуемые оценками точности. Эти функции можно
описать как функции, которые сравнивают результирующее
множество с тестовым и определяют качество как точность \lq попадания \rq
объекта из результирующего множества в тестовое.
Эти функции заимствованы из области информационного поиска \cite{ir1,ir2,ir3}.
При оценке задачи $topN$ состоит из объектов, между которыми и активным
пользователем выполняется отношение близости, как и при решении задачи:
$P^a_{\bot} = \{ (u_a,  i_{\bot}, \rho(u_a, i_{\bot})): u_a \R i_{\bot}\}$.

Для определения используемых оценок точности введем функцию,
которая определяет, выполняется ли отношение $u_a \R i$:
$
s(i) =
\begin{cases}
1, &\text{$\exists$ $i_{\bot} \in I_{\bot}$: $u_a \R i_{\bot}$}\\
0, &\text{иначе}.
\end{cases}
$\\

Функция $s$, заданная таким образом соответствует цели пользователя, поэтому
оценки точности, которые определены на базе такой функции $s$ будем называть
{\it целевыми}.

Приведем пример целевых оценок точности:
\begin{itemize}
\item Точность:
	\begin{equation}
		\label{precision}
	1 - \frac{1}{N} \cdot \sum \limits_{i \in I_{\bot}} s(i);
	\end{equation}
%\item Полнота: $1 - \frac{1}{M} \cdot \sum \limits_{i \in I_{\bot}} s(i)$,
%где $M = \{|i \in \mathbb{N}^n : a \R i |\}$ ---
%число всех объектов множества $I$, для которых выполняется отношение
%близости с активным пользователем. Как правило, полнота оценивается на
%объединении результирующих множеств, полученных на большом числе тестов
%или при $N \approx M$. Будем считать, что для полноты $M = N$\label{recall-meqn};
\item Точность $P@L$ для списка объектов длины $L$:
\begin{equation}
1 - \frac{1}{L} \cdot \sum \limits_{i \in I_L} s(n),I_L \subset I_{\bot},
|I_L| = L.
\end{equation}
\item
	\begin{equation}
	AveP = \frac{1}{\sum \limits_{n=1}^{N} s(n)} \cdot
\sum \limits_{L=1}^{N},
\end{equation}
		где $P@L$ --- среднее значение $P@L$ для $L=1..N$.
	\item
		\begin{equation}
		NDCG = 1 - \frac{DCG}{IDCG},
		\end{equation}
			где $DCG = s(1) + \sum \limits_{k=1}^N
\frac{s(i_k)}{log_2(i_k)}$ --- сумма <<весов накопления>>,
где <<вес накопления>> зависит от порядкового номера
в ранжированном результирующем множестве.
Чем меньше порядковый номер объекта,
между которым и активным пользователем выполняется отношение
близости, тем результат качественней, а вес объекта больше.
		Данная оценка используется, к примеру, для поисковых систем\cite{},
		где для пользователя важно, чтобы интересующий его объект находился в
		начале результирующего списка.
\end{itemize}

Однако, совместно с ООМ эти оценки не могут использоваться, так как ООМ
не определяют функцию $\rho: U \times I$. Вместо нее используются значения
прогнозной функции, которые рассчитываются на основании правил вывода ООМ.
Подобно решению, оценка решения задачи $topN$ при применении ООМ,
основана на эвристическом утверждении (\ref{assertORS1}). Чтобы определить,
является ли объект $i \in I_{topN}$ целевым по отношению к поставленной задачи,
нужно выяснить, существует ли объект обучающего множества, являющийся соседом
для объекта $i$. Так как для объектов обучающего множества верно, что $u_a \R
i_0$, то по эвристическому утверждению (\ref{assertORS1}) получим, что $u_a \R
i$, так как $i \rt i_0$.

Для определения качества решения в ООМ используются те же целевые функции,
но их значения зависят не от функции $s$, а от функции $\overline{s}$:
$
\overline{s}(i) =
\begin{cases}
	1, &\text{$\exists$ $i_{\bot} \in I_{\bot}$: $i \rt i_{\bot}$}\\
0, &\text{иначе}.
\end{cases}
$\\
Такие оценки точности будет называть {\it объектно-ориентированными}.
