\section{Обобщение оценок качества}
Анализ качества решений будет
производится по отношению к определенным выше оценкам качества
решений (\ref{def-eval}).
Было показано, что существует два класса оценок, каждому из которых
принадлежит некоторое число функций. Прежде, чем приступить к рассмотрению
качества решений, проведем обобщение функций,
принадлежащих каждому классу,
и будем работать далее с обобщенной функцией класса, что возможно
сделать в силу того, что оценки, которые принадлежать одному классу коррелируют
между собой \cite{herloker-eval}.

%\subsection{Истинность правил вывода}
%Рассмотрим правила вывода и эвристические утверждения, на которых они основаны.
%Выполнение приведенных в первой главе эвристических
%утверждений (\ref{}) () () в существующих исследованиях не рассматриваются,
%правила вывода, основанные на них носят аксиоматический характер.
%\subsection{Условия эффективности решений}
%Прежде, чем приступить к дальнейшему описанию, обобщим оценки
%решения и в дальнейшем для изложения материала будем работать с обобщенной
%оценкой. Возможно провести обобщение и использовать одну функцию
%при описании свойств нескольких, так как оценки, которые принадлежат одному
%классу, коррелируют друг с другом.
% \cite{herloker-eval}.

\subsubsection{Обобщенная оценка качества решения задачи прогнозирования}
По задаче $pred$ необходимо вычислить такое значение
$\rh(u_a, i_{\bot})$, что $|\rh(u_a, i_{\bot}) - \rho(u_a, i_{\bot})| \le
\varepsilon_p$.
Оценки качества решения задачи $pred$
определяют погрешность
составления прогноза, то есть все функции данного класса зависят от одного и того же параметра: от
разницы между спрогнозированной оценкой близости и настоящей.
Чем меньше разница, тем значение оценки меньше и тем аккуратность выше. Функции, принадлежащие этому
классу коррелируют между собой  \cite{herloker-eval}. Поэтому можем ввести
обобщенную функцию и работать в дальнейшем с ней.
Введем и обозначим следующую обобщенную оценку
для этого класса: $\eap = NMAE$.

\begin{assert}
	\label{objectivity-eap}
	Оценка $\eap$ соответствует задаче $pred$, а потому ее значения
	являются объективным показателем аккуратности решения.
\end{assert}
Данное утверждение следует из постановки задачи и способу определения
аккуратности.
\subsection{Обобщенная оценка качества решения задачи $topN$}
Значения приведенных в предыдущем разделе оценок качества задачи $topN$ зависит
от числа
$K = |I'| = |\{i: (i \in I_{topN}) \wedge (s'(i) = 1)\}|$,
где $s' = \{s, \overline{s}\}$.

\begin{trm}
	Решение задачи $topN$ эффективно по оценке качества, если $K \ge N \cdot (1 - \varepsilon_{topN})$
\end{trm}

Покажем, что теорема верна для каждой оценки, приведенной выше \ref{def-eval}
на примере точности (а оценки качества одного класса коррелируют между собой).
%\begin{itemize}
%\item
%	Точность:
		$\eit \le 1 - \frac{K}{N} \le 1 - (1 - \varepsilon_{topN}) \le
		\varepsilon_{topN}$.
%\item Полнота: так как рассматривается при $N = M$\ref{recall-meqn},
%	то доказательство сводится к доказательству, относящемуся к  точности.
%\item Точность $P@L$ --- верно, так как данная оценка является частным случаем
%	точности. Точность может рассматриваться как $P@N$.
%Если теорема верна для $P@N$, то она верна и для $P@L$, так как $L \le N$.
%\item $AveP$: так как отношение близости выполняется для $i_k, k=1..K$,
%то $P@L = 0, l=1..K$ и
%$\eat = \frac{1}{K} \cdot \sum \limits_{L=K}^{N} P@L$.
%В худшем случае, когда $P@L=1, l=(K+1)..N$ получим, что:\\
%$\eat = \frac{N-K}{K} = \frac{N}{K} - 1 \le \frac{N}{N \cdot (1 - \varepsilon_{topN})}
%		- 1 = \frac{1}{1 - \varepsilon_{topN}} - 1$.
%В пределе при $\varepsilon_{topN} \rightarrow 0$ ($K \rightarrow N$)
%		решение эффективно для любых значений $\varepsilon_{topN} \in \varepsilon(0)$.
%
%\item $IDCG = 1 + \sum \limits_{k=2}^N \frac{1}{log_2(k)}$. Так как $s(i_k)=1,
%	k=1..K$, то $DCG=1 + \sum \limits_{n=2}^K \frac{1}{log_2(n)}$.
%$NDCG = 1 - \frac{1 + \sum \limits_{k=2}^K \frac{1}{log_2(n)}}{1 + \sum
%		\limits_{k=2}^N \frac{1}{log_2(k)}}$.
%В пределе при $K \rightarrow N$ получим, что $NDCG \rightarrow 0$, то есть решение эффективно.
%\end{itemize}

Таким образом, мы показали, что приведенные оценки зависят от величины $K$.
Тогда будем рассматривать обобщенную целевую оценку качества решения задачи
$topN$ как функцию $\eat(K)$ при применении функции $s$,
и обобщенную объектно-ориентированную оценку $\eit(K)$ при применении
$\overline{s}$.

