\chapter*{Введение}
\addcontentsline{toc}{chapter}{Введение}
{\bf Актуальность исследования и степень разработанности}.
С интенсивным развитием электронной коммерции, социальных сетей,
мобильных устройств, имеющих доступ к интернету и мобильных приложений,
разрабатываемых для них, вырос спрос на веб-сервисам
и информационным системам, в функциональность которых
входят рекомендательные системы.
Эти системы облегчают
пользователю задачу
поиска нужной ему информации путем рекомендаций.
%или путем определения степени близости конкретной информации пользователю.

Известными примерами веб-сервисов, в функциональность которых входит
рекомендательный сервис, являются: Netflix \cite{Netflix}
и Youtube \cite{youtube} ---
включают в себя кинематографический рекомендательный сервис,
Google News и Yahoo! News ---
включают в себя новостной рекомендательные сервис,
Amazon и Ebay ---
включают в себя товарный рекомендательные сервис.
Подобные сервисы
предоставляют пользователям огромное число интересующей их информации.
Например, кинематографический веб-сервис Netflix имеет в своей базе свыше
17000 фильмов, Amazon.com свыше 410000 наименований товаров.

Для рекомендательных сервисов характерно решение следующих двух задач:
\begin{enumerate}[label*=\arabic*)]
	\item быстрое предоставление пользователю списка такой информации заданной длины $N$,
		которая будет близка предпочтениям, вкусам потребностям и т.п.
		пользователя;

	\item определение степени близости конкретной информации к предпочтениям,
		вкусам, потребностям и т. п. пользователя.
\end{enumerate}

Первая задача называется задачей $topN$, где $N$ --- длина списка,
вторая --- задачей прогнозирования $pred$.
Способы решений задач $topN$ и $pred$
рассматриваются в отдельном направлении области компьютерных
наук, именуемом рекомендательные системы (далее РС). Это направление
берет свои истоки из информационного поиска, машинного обучения и теории
принятия решений. Начиная с 2007, ежегодно проводятся международные конференции
Rec Sys (от Recommender Systems), посвященные РС. Данные конференции
проводятся под эгидой Ассоциации вычислительной техники ACM.
В 2014 году было опубликовано свыше 170 работ, посвященных РС \cite{number-of-researches-2014}
%с высоким импакт-фактором.
Эти данные подтверждают не только коммерческий, но и академический интерес к
РС.

Существуют различные модели РС, Которые задают способ
представления данных РС и техники решения задач.
Одной из самых
популярных, разработанных и коммерчески успешных моделей
является коллаборативная модель (далее АКМ) \cite{most-researched, rs-in-compsciense}
Эти модели используют технику коллаборативной фильтрации.
КМ разбиваются на два класса: анамнестические и модельные. В исследовании
будут анализироваться и оцениваться анамнестические АКМ (далее АКМ).
Данные методы отфильтровывают множество пользователей или множество объектов
на основании значений функций, именуемых мерами близости.
Оценочную характеристику РС будем
называть эффективностью.

Эффективность модели РС будем определять по следующим критериям:
\begin{enumerate}[label*=\arabic*)]
\item качество решения. Показателем эффективности по критерию качества
	является некоторое число, являющееся значением специальной функции,
	именуемой оценкой качества и обозначаемой $\mathcal{E}_t$, где
	$t \in \{topN, pred\}$.
	Часть из используемых функций в роли оценки качества заимствованы из
	таких областей как информационный поиска и математическая статистика;
\item вычислительная сложность алгоритма решения. От вычислительной сложности
	зависят такие важные показатели системы, как масштабируемость и
	производительность. Вычислительную сложность будем характеризовать
	асимптотической сложностью алгоритмов;
\item стабильность. Стабильностью будем называть свойство системы, которое
	заключается в способности РС формировать эффективное решение по критерию
	качества
	вне зависимости от
	свойств исходных данных.
\end{enumerate}
Будем говорить, что модель РС эффективна по некоторому критерию, если она
удовлетворяет ему независимо от дополнительных условий и ограничений.

Построение эффективной модели по заданным критериям является
задачей сложной и актуальной. К примеру, компания Netflix
проводила конкурс по созданию алгоритма формирования качественного решения
задачи прогнозирования, основанного на коллаборативной фильтрации, а главный
приз равнялся миллиону долларов.
Эффективный веб-сервис может увеличить доход компании,
которая его использует для продвижения товаров, а потому,
существует коммерческий интерес к развитию рекомендательных систем и сервисов.

АКМ внедрены
во многие известные веб-сервисы: Amazon, Netflix, IMDB, Kinopoisk, LastFm и т.д.
Изучением коллаборативной фильтрации занимались такие известные исследователи как
Дж. Констан \cite{cfrs, usenet, item-based,framework-cf,e-commerce,content8,cf-expert},
Г. Ф. Рикки \cite{heur3,rs-handbook},
Б. Сарвар \cite{item-based,e-commerce},
А. Тужилин \cite{heur1,toward,2d},
Дж. Карипис \cite{item-based,e-commerce,topn1,topn2},
Дж. Херлокер \cite{usenet,framework-cf,content8,cf-expert,cluster1},
и др.
Несмотря на успешность, популярность и заявляемую разработанность
КМ и на то,
что подобные системы были интегрированы в бизнес более десяти лет назад,
а их методы уже стали называться А. Тужилиным, М. Экстрандом,
Дж. Ридлом и Дж. Констаном традиционными \cite{cfrs, 2d},
существует ряд открытых актуальных проблем, связанных с применением АКМ.

Методы коллаборативной фильтрации основаны на эвристических утверждениях,
что служит причиной, по которой АКМ не гарантируют в общем случае
получения эффективного решения по критерию стабильности.
По этой же причине в исследованиях, например,
А. Тужилина, Ф. Рикки, Дж. Кариписа и Дж. Констана, АКМ называются
эвристическими \cite{heur3,heur1, heur2}.
Выполнимость эвристических утверждений не рассматривается в существующих
исследованиях, однако, как показано в диссертационной работе, их выполнение
не всегда возможно и зависит от свойств исходных данных,
поэтому АКМ не является эффективной моделью
по критерию стабильности.

В диссертационном исследовании определены
достаточные условия эффективности АКМ по критерию качества решения
Выполнение этих условия зависит параметров АКМ, выбор которых зависит
от разработчиков РС. Причем не всегда можно подобрать параметры так, чтоб
выполнялись достаточные условия и РС удовлетворяла требованиям заказчика.
Таким образом, эффективность РС по критерию качества ограничены дополнительными
условиями, и поэтому АКМ не является эффективной моделью по критерию качества.

Современные РС работают с огромным числом пользователей и объектов.
Асимптотические сложности алгоритмов решений АКМ таковы, что, учитывая
большое число объектов и пользователей, проблема масштабируемости
имеет место быть
при применении АКМ \cite{amazon-item2item,amazon-linden}.

%На данном этапе развития области РС не существует единой терминологии и
%о чем свидетельствует, к примеру, терминология различных исследований,
%используемая для описания методик решений, основанных на коллаборативной
%фильтрации, где фильтрация производится по множеству объектов:
%item-based или item-to-item рекомендации \cite{item-based,
%topn1,topn2,heur2,amazon-item2item,amazon-linden,item-based-cross-sell},
%модельная техника решения
%(model-based)
%\cite{topn1,item-based-cross-sell,ringo,learning4,model-based1,empirical-cf},
%контентная техника решения (content-based) \cite{content1,content10}.

%Существуют эмпирические исследования, которые устанавливают корреляцию между
%эффективностью по критерию качества, и тем, какие функции использовались для
%вычисления степени сходства пользователей или объектов
% \cite{content_rs_soc_sys, eval-sim-metrics}.
%В таких исследованиях демонстрируется, что на одних
%данных применение одних функций дают более эффективный результат, чем на других.
%На других данных --- наоборот. Однако подобные эмпирические выводы не могут
%быть расширены и применены для любых исходных данных, что остается без внимания.

АКМ могут применяться для решения задач только в тех случаях,
когда существует множество данных, содержащее инфоромацию о предпочтениях,
вкусах, потребностях и т.п. пользователей по отношению к объектам.
Эти данные берутся
из истории поведения пользователя в системе (например, история покупок) или
могут быть заданы самим пользователем (например, путем задания оценки
конкретным объектам). Если такие данные отсутствуют или их объем мал,
то АКМ не могут быть применены. Отсюда следует наличие таких проблем, как
холодный старт, когда решение не может быть предоставлено новому
пользователю РС, или новые объекты не попадают в список рекомендуемых объектов.

%Для оценки решения задачи по критерию качества вычисляются функции,
%именуемые оценками. Каждой задаче соответствует
%отдельный класс оценок. Между оценками,
%принадлежащими разным классам, отсутствует корреляция, а в каждый класс
%входит порядка десятка функций оценок. В связи с этим,
%в исследованиях Дж. Херлокера, Дж. Констана и Дж. Ридла
% \cite{framework-cf, herloker-eval}
%поднимается проблема выбора оценки, и проблема сравнения
%эффективности по критерию качества, которое
%было определено разными оценками и при решении разных задач.
%Данная проблема определяется как проблема отсутствия стандарта оценки.

%Основная гипотеза работы заключается в следующем: теория АКМ слабо
%формализована, следствием чего являются открытые проблемы, и, в общем
%случае, АКМ не являются эффективными по всем трем приведенным критериям.
%Повышение эффективности возможно при соблюдении выполнения
%определенных условий. Приводится альтернативная формальная модель РС, не
%основанная на эвристических утверждениях и обладающая
%большей эффективностью по сравнению АКМ.

%ш^том, что если формально
%техники решения, провести их анализ, определить условия, влияющие на
%эффективность РС, то можно реформировать существующие коллаборативные
%техники так, что эффективность АКМ повысится. Помимо реформации существующих
%техник возможно создание такой математической модели РС, которая будет являться
%эффективным расширением существующих АКМ. Для того, чтобы построить такую
%модель, нужно прибегнуть к подходящим математическим аппаратам, которые будут
%заложены в основу РС, без применения эвристик. В качестве такого аппарата была
%выбрана теория нечетких множеств, на которой была построена математическая
%модель РС. Построенная модель, как показана в исследовании более эффективна,
%чем АКМ.

%В диссертационном исследовании рассматриваются следующие
%открытые проблемы:
%\begin{itemize}
%	\item отсутствие единой формальной теории, терминологии и обозначений АКМ;
%	\item исследования, в основном, носят эмпирический характер, анализ
%		причин, почему один и тот же метод хорошо работает на одних данных, а
%			на других --- нет, отсутствует;
%	\item алгоритмы решения задач основаны на эвристических утверждения,
%		анализ выполнения которых не производится;
%	\item эффективность АКМ ограничивается дополнительными условиями:
%		\begin{itemize}
%			\item получение эффективного решения по критерию качества
%				ограничено его зависимостью от выбора функции, используемой
%				в качестве меры сходства агентов, и порогового значения
%				этой функции.
%				Выбор этих
%				параметров АКМ влияет на выполнение свойства транзитивности
%				близости агентов, что является достаточным условием получения
%				эффективного решения по критерию качества. Выбор параметров
%				производится разработчиками РС, поэтому за получение
%				эффективного решения по критерию качества ответственны
%				разработчики, а в общем случае для любых функций и пороговых
%				значений АКМ неэффективны по критерию качества;
%			\item получение эффективного решения по критерию стабильности
%				ограничено свойствами исходных данных.
%				Если исходные данные таковы, что
%				для них характерно свойство динамики или гетерогенности, то
%				эвристические утверждения не выполняются, а потому нет гарантии
%				получения решения, эффективного по критерию качества. Таким
%				образом, в общем случае применение АКМ не гарантирует получения
%				эффективного решения по критерию качества на любых исходных
%				данных;
%			\item получение эффективного решения по критерию вычислительной
%				сложности ограничено асимптотической сложностью алгоритмов
%				решения задач.
%		\end{itemize}
%	\item оценки, используемые для
%		определения качества решения задачи $topN$, не являются объективными
%		показателями в случае, если исходные данные обладают свойством
%		гетерогенности.
%\end{itemize}

{\bf Цель диссертационного исследования} --- разработать
формальную математическую модель РС, являющуюся эффективным расширением
АКМ и определить в разработанной модели
методы решения задач более эффективные, чем методы АКМ.

Для достижения цели были поставлены следующие
{\bf основные задачи}:
\begin{itemize}
	\item
		провести анализ эффективности АКМ по критериям оценки
		эффективности моделей РС. Показать, что в общем случае
		АКМ не является эффективной моделью РС;
	\item разработать модель рекомендательной системы на нечетких
		множествах и определить такие алгоритмы решений задач в разработанной
		модели, эффективность которых по критериям качества, стабильности и
		вычислительной
		сложности будет выше алгоритмов коллаборативной модели для любых
		исходных данных.
		Провести теоретическое сравнение разработанной модели и
		коллаборативной по определенным критериям;
	\item разработать программное обеспечение, с помощью которого
		провести тестирование, в ходе которого решить задачи стандартными
		методами коллаборативной фильтрации и методами, определенными в
		нечеткой модели. Сравнить полученные результаты.
\end{itemize}

{\bf Научная новизна} полученных в диссертационной работе результатов:
\begin{enumerate}
\item впервые сформулированы достаточные условия, при выполнении которых гарантируется,
	что решения задач, полученные при применении АКМ,
	будут эффективными по критерию качества;
\item разработана оригинальная математическая модель РС, основанная не теории
	нечетких множеств. Разработанная модель
	является эффективным расширением АКМ и
	позволяет новым способом использовать доступную в современных
	условиях контекстную информацию о пользователях;
\item впервые определено отображение метаданных пользователя
	на множество метаданных объектов, которое используется для решения задач.
\end{enumerate}

{\bf Результаты, выносимые на защиту}:
\begin{itemize}
\item результаты анализа АКМ:
  \begin{itemize}
	\item достаточное условие, при выполнении которого гарантируется получение
		эффективного решения задачи $topN$ по критерию качества;
	\item достаточное условие, при выполнении которого гарантируется получение
		эффективного решения задачи $pred$ по критерию качества;
  \end{itemize}
\item математическая модель РС, являющаяся эффективным расширением АКМ;
%\item алгоритм решения задачи $pred$ методами коллаборативной
%	фильтрации в разработанной модели, для которого выполняется
%	выведенное достаточное условие;
\item методика формального задания взаимосвязи между информацией
	о пользователе и объекте,
	заключающийся в определении отображения пользователя на множество объектов,
	за счет которого обеспечивается вычисление прогнозной функции как
	расстояния между пользователем и объектом;
\item алгоритмы решения задач, основанные на использовании заданного
	расстояния между пользователем и объектом;
\item разработанное программное обеспечение, с помощью которого проводилось
	тестирование моделей, и сравнительный анализ полученных
	результатов тестирования.
\end{itemize}


{\bf Практическая значимость работы} заключается в использовании
полученных теоретических результатов для практической программной реализации
эффективной РС, которая может быть применена не только в таких стандартных
областях применения РС, как, например, интернет-магазины,
где известно предпочтение пользователя по отношению к объекту, но и
в областях, где можно определить более сложные взаимосвязи на основании
контекстной информации о пользователях и объектах \cite{2d}.
Полученные результаты могут быть также использованы для повышения
эффективности уже существующей программной реализации АКМ.

{\bf Апробация результатов работы.}
Основные результаты диссертации докладывались и обсуждались на следующих научных мероприятиях:
\begin{enumerate}
\item XII международная конференция <<Russian Conference on \\Digital Libraries>>
	(г. Казань, 2010 г.);
\item Молодежная школа-семинар <<Модели и методы исследования систем
	структуры>>, (пос. Дивноморское, 2011, 2014);
\item Международная конференция <<Управление и оптимизация неголономных
	систем>>,(г. Переславль-Залесский, 2013);
\item Ученый совет кафедры прикладной математики и информатики
	НИУ ВШЭ (г. Нижний Новгород, 2016);
\item Ученый совет Исследовательского центра искусственного интеллекта
	ИПС им. А. К. Айламазяна РАН (г. Переславль-Залесский , 2016);
\item Семинар компании IT-Aces (г. Переславль-Залесский, 2016);
\item XIX международная конференция <<Data Analytics and Management in Data
	Intensive Domains>> (г. Москва, 2017)
\end{enumerate}

{\bf Публикации.} Основные научные результаты по теме диссертации опубликованы
в 3 печатных журналах, 2 из которых рекомендованы ВАК РФ, и 5 электронных
журналах, 3 из которых рекомендованы ВАК РФ и 1 включен в БД Scopus.

{\bf Личный вклад автора.} Автору принадлежат содержащиеся в
диссертации результаты теоретического исследования АКМ,
разработанная математическая модель на нечетких множествах,
алгоритмы решений задач, определенные в разработанной модели,
способы повышения эффективности коллаборативных методов в разработанной модели
и программное обеспечение для проведения тестирования.

{\bf Реализация и внедрение результатов работы.}
В ходе диссертационного
исследования было написано программное обеспечение на
языке программирования С++ для проведения тестирования и с возможностью
его дальнейшего применения в реальных проектах в качестве модуля,
обеспечивающего решение задачи $topN$ и $pred$ в различных моделях и
при применении различных алгоритмов. С помощью программного обеспечения
было проведено тестирование, на основании результатов которого
было проведено практическое сравнение разработанной модели и АКМ.

Разработанное программное обеспечение получило свидетельство
о государственной программы для ЭВМ – $\textnumero$ 2013612828 <<Контентный
рекомендательный сервис>>.

Разработанное программное обеспечение было представлено
на семинаре компании <<IT-Aces>>.

Результаты исследования нашли применение в компании
<<IT-Aces>> в процессе проектирования внутренних веб-сервисов, что
подтверждается актом о внедрении.

%Опубликованные статьи:
%		\begin{enumerate}
%		\item Понизовкин Д. М.
%		Оптимальное распределение проектов при проведении экспертизы / Д. М. Понизовкин, С. А. Амелькин //
%		Электронные библиотеки: Перспективные Методы и Технологии,
%		Электронные коллекции. --- 2010. --- С. 524-525.
%		% электронный, ВАК
%		\item Понизовкин Д. М. Построение оптимального графа связей в системах коллаборативной фильтрации / Д. М. Понизовкин, С. А. Амелькин //
%		Программные системы: теория и приложения. 2011.--- Т. 2. --- № 4. С. 107–114
%		% электронный, ВАК
%		\item Понизовкин Д. М. Математическая модель коллаборативных процессов принятия решений //
%		Программные системы: теория и приложения. 2011. --- Т. 2. --- № 4. С 95-99.
%		% электронный
%		\item Амелькин С. А, Д. М. Понизовкин. Оптимальное проведение экспертизы образовательных процессов / С. А. Амелькин, Д. М. Понизовкин //
%		Труды XVII Всероссийской научно-методической конференции Телематика’2010,
%		Санкт-Петербург: Университетские телекоммуникации. --- 2010. ---
%		№ 1, С. 158-159.
%		% электронный, ВАК
%		\item Д. М. Понизовкин. Влияние меры сходства на результативность РС // Программные системы: теория и приложения, 2014. --- т. 2. --- N. 5. С 55–65.
%		% печатный, ВАК
%		\item С. А. Амелькин, Д. П. Понизовкин. Математическая модель задачи $topN$ для
%			контентных рекомендательных систем //
%		Известия МГТУ МАМИ, 2 , c. 26–31
%
%		% Печатный, ВАК
%		\item Д. М. Понизовкин.
%			Повышение качества решения задачи $topN$ коллаборативными рекомендательными
%			системами
%			//
%			Современная наука: актуальные проблемы теории и практики.
%			Серия естественные и технические науки. 2017 ---
%			Т. 7-8. --- с 62-67.
%		% Электронный web of science
%		\item Д. М. Понизовкин.
%				Модель рекомендательной системы на нечетких множествах
%				как эффективное расширение коллаборативной модели//
%				Data Analytics and Management in Data Intensive Domains:
%				Collection of Scientific Papers of the XIX International
%				Conference DAMDID / RCDL’2017 (October 10–13, 2017, Moscow, Russia),
%				2017. - c 118-123/
%		\end{enumerate}

{\bf Структура и объем диссертации}. Диссертация состоит из введения, пяти глав
и заключения, списка литературы и приложения. Материал изложен на 162
страницах, содержит 18 таблиц, 30 рисунков, 110 литературных источников и 2
приложений.

{\bf Во  введении} аргументирована  актуальность  темы  диссертационного
исследования, представлена степень ее проработанности, сформулированы цель и
задачи  исследования,  рассмотрены  объект,  предмет  и  методы  исследования,
отражены научная новизна, теоретическая и практическая значимость результатов,
приведены сведения о внедрении и использовании результатов.

{\bf В первой главе} заданы основные термины и обозначения, на основании
которых определены модель РС, модели АКМ, критерии эффективности
моделей, задачи, основные методы решений и способы оценки качества решений.

{\bf В второй главе} проведен анализ описанных в первой главе методов решений
и способов оценки. Определены основные проблемы, условия и ограничения АКМ.

{\bf В третьей главе} описана разработанная математическая модель на основе
нечетких множеств и методы решений задач в этой модели. Проведено теоретическое
сравнение разработанной модели с АКМ и показано, что разработанная модель
является эффективным расширением АКМ.


{\bf В четвертой главе} описано программное обеспечение, разработанное для
тестирования, множество входных данных, применяющееся для
тестирования, методы тестирования, результаты тестов. Проведен практический
сравнительный анализ разработанной модели и АКМ на основании полученных тестов.

{\bf В пятой главе} описано веб-приложение рекомендательного веб-сервиса,
алгоритмы которого основаны на применении разработанной модели.
