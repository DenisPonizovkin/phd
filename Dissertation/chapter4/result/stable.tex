
\section{Сравнение моделей по критерию стабильности}
\subsection{Влияние свойства динамики исходных данных}

В аналитической главе говорилось о влиянии свойства динамики
данных (\ref{ass:dynamic}) на качество решения в СОМ, а точнее на выполнение
эвристического утверждения СОМ (\ref{srs-assert}). Для теста данного пункта
была произведена имитация проявления свойства динамики, для этого
исходные данные были разбиты не стандартно в пропорции 80 к 20, а
в пропорции 40 к 60, то есть в обучающее множество попало 40\%
данных, остальные --- в тестовое. При таком разбиении вероятность
того, что $u \ru v$ на обучающем множестве выполняется, но на тестовом --- нет,
высока.


Результаты представлены в следующих формах:
\begin{enumerate}
	\item таблицей <<Влияние свойства динамики на качество решения при применении
	$\Pi_C$ в СОМ>> (\ref{tbl:dynamic-com}), в которой указаны
	средние значения $NMAE$ при решении задачи прогнозирования
	в СОМ и при использовании стандартного алгоритма
	на стандартно разбиении и на разбиении 40/60. Значения
	$NMAE$ для разбиения 40/60 выше, что говорит о зависимости
	качества решения от выполнения эвристического утверждения и его
		зависимости от свойств исходных данных;

	\item таблицей <<Влияние свойства динамики на качество решения при применении
	$\Pi_C$ в нечеткой модели>> (\ref{tbl:dynamic-com}), в которой указаны
	средние значения $NMAE$ при решении задачи прогнозирования
	в нечеткой модели и при использовании стандартного алгоритма
	на стандартно разбиении и на разбиении 40/60. Значения
	$NMAE$ для разбиения 40/60 выше, что говорит о зависимости
	качества решения от выполнения эвристического утверждения и его
	зависимости от свойств исходных данных независимо от используемой модели;

	\item таблицей <<Влияние свойства динамики на качество решения при применении
	$\Pi_f$ в нечеткой модели>> (\ref{tbl:dynamic-fuz-com}), в которой указаны
	средние значения $NMAE$ для стандартного разбиения и 40/60
	при решении задачи прогнозирования в нечеткой модели и при использовании
	алгоритма $\Pi_f$, основанного на $\Pi_f$.
	В таблице (\ref{tbl:dynamic-fuz-com}) среднее значение $NMAE$, соответствующее
	решению задачи на разбиении 40/60 больше, чем среднее значение $NMAE$
	на стандартном разбиении
	что свидетельствует о влиянии свойства динамики на качество решения
	при использовании $\Pi_f$ в нечеткой модели. Проявление влияния следует из
	того, что функция $\delta_c$ составлена по данным, которые принадлежат
	множеству $P$. Но, несмотря на влияние, эффективность нечеткой модели
	при применении $\Pi_f$
	снижается не столь стремительно как в случае применения $\Pi_C$
	и дает в условиях динамики более эффективный результат,
	что подтверждает теоретический вывод о том, что нечеткая
	модель более эффективна по критерию стабильности.
\end{enumerate}

\begin{table}[h]
	\caption{Влияние свойства динамики на качество решения при применении
	$\Pi_C$ в СОМ}
  \begin{center}
	\label{tbl:dynamic-com}
	\begin{tabular}{|c|c|}
	  \hline
		Разбиение & NMAE \\ \hline
		80/20 & 0,495 \\ \hline
		40/60 & 0,634 \\ \hline
	\end{tabular}
  \end{center}
\end{table}

\begin{table}[h]
	\caption{Влияние свойства динамики на качество решения при применении
	$\Pi_C$ в нечеткой модели}
  \begin{center}
	\label{tbl:dynamic-fuz-com}
	\begin{tabular}{|c|c|}
	  \hline
		Разбиение & NMAE \\ \hline
		80/20$\Pi_{O}$&0,001 \\ \hline
		40/60$\Pi_{f}$&0,420 \\ \hline
	\end{tabular}
  \end{center}
\end{table}

\begin{table}[h]
	\caption{Влияние свойства динамики на качество решения при применении
	$\Pi_f$ в нечеткой модели}
  \begin{center}
	\label{tbl:dynamic-fuz}
	\begin{tabular}{|c|c|}
	  \hline
		Разбиение & NMAE \\ \hline
		80/20$\Pi_{O}$&0,334 \\ \hline
		40/60$\Pi_{f}$&0,398 \\ \hline
	\end{tabular}
  \end{center}
\end{table}
Практические результаты подтверждают теоретические выводы о том,
что нечеткая модель при применении $\Pi_f$ более эффективна по критерию
стабильности (\ref{trm:fuz-eff-extension-stab}).

% hetero
\subsection{Влияние свойства неоднородности}
В аналитической главе говорилось о влиянии свойства неоднородности данных
(\ref{ass:hetero}) на качество решения в ООМ.
Свойство неоднородности в используемых для тестов данных
проявляется для некоторых пользователей. Это проявление
заключается в том, что пользователь высоко оценивает фильмы,
которые не схожи друг с другом по характеристикам. Например,
пользователь высоко оценивает фильм жанра <<comedy>> и фильм
жанра <<crime>>.

В данном пункте будут представлены результаты теста, которые
проводились на подмножестве таких пользователей $U^{\prime}$,
для которых проявляется свойство неоднородности в большей мере,
то есть процент объектов, между которыми не выполняется отношение близости
составляет больше 70\%.

Результаты представлены в следующих формах:
\begin{enumerate}
	\item таблицей <<Влияние неоднородности на ООМ>> (\ref{tbl:hetero-oom}), в которой указаны
	средние значения точности для множества пользователей $U$ и подмножества
	$U^{\prime}$ при решении задачи с помощью ООМ.
	В данной таблице (\ref{tbl:hetero-oom}) среднее значение точности, соответствующее
	решению задачи на подмножестве $U^{\prime}$, на порядок
	меньше значения точности решения для всего множества,
	что свидетельствует о влиянии свойства неоднородности на качество
	решения при использовании $\Pi_O$ в ООМ.

	\item таблицей <<Влияние неоднородности на нечеткую модель при применении $\Pi_O$>> (\ref{tbl:hetero-fuz-oom}), в которой указаны
	средние значения точности для множества пользователей $U$ и подмножества
	$U^{\prime}$ при решении задачи с помощью $\Pi_O$ в нечеткой модели.
	В данной таблице (\ref{tbl:hetero-fuz-oom}) среднее значение точности, соответствующее
	решению задачи на подмножестве $U^{\prime}$, на порядок
	меньше значения точности решения для всего множества,
	что свидетельствует о влиянии свойства неоднородности на качество решения
	при использовании $\Pi_O$ в нечеткой модели. В ООМ и нечеткой модели
	использовался один и тот же алгоритм, эффективность которого заметно
	снижается при проявлении свойства неоднородности, то есть независимо
	используемый алгоритм не дает эффективного результата.

	\item таблицей <<Влияние неоднородности на нечеткую модель при применении $\Pi_f$>> (\ref{tbl:hetero-fuz-oom}), в которой указаны
	средние значения точности для множества пользователей $U$ и подмножества
	$U^{\prime}$ при решении задачи с помощью $\Pi_O$ в нечеткой модели.
	В данной таблице (\ref{tbl:hetero-fuz-oom}) среднее значение точности, соответствующее
	решению задачи на подмножестве $U^{\prime}$, на порядок
	меньше значения точности решения для всего множества,
	что свидетельствует о влиянии свойства неоднородности на качество решения
	при использовании $\Pi_f$ в нечеткой модели. Проявление влияния следует из
	того, что функция $\delta_c$ составлена по данным, которые принадлежат
	множеству $P$. Но, несмотря на влияние, эффективность нечеткой модели
	при применении $\Pi_f$
	снижается не столь стремительно как в случае применения $\Pi_O$
	и дает в условиях неоднородности более эффективный результат,
	что подтверждает теоретический вывод о том, что нечеткая
	модель более эффективна по критерию стабильности.
\end{enumerate}

\begin{table}[h]
	\caption{Влияние неоднородности на ООМ}
  \begin{center}
	\label{tbl:hetero-oom}
	\begin{tabular}{|c|c|}
	  \hline
		Множество & Точность \\ \hline
		Все пользователи & 0,251 \\ \hline
		только те, для которых неоднородность выполняется & 0,108 \\ \hline
	\end{tabular}
  \end{center}
\end{table}

\begin{table}[h]
	\caption{Влияние неоднородности на нечеткую модель при применении $\Pi_O$}
  \begin{center}
	\label{tbl:hetero-fuz-oom}
	\begin{tabular}{|c|c|}
	  \hline
		Модель/Правило вывода & Точность \\ \hline
		ООМ/$\Pi_{O}$&0,644 \\ \hline
		Нечеткая/$\Pi_{f}$&0,277 \\ \hline
	\end{tabular}
  \end{center}
\end{table}

\begin{table}[h]
	\caption{Влияние неоднородности на нечеткую модель при применении $\Pi_f$}
  \begin{center}
	\label{tbl:hetero-fuz}
	\begin{tabular}{|c|c|}
	  \hline
		Свойство неоднородности проявляется & Точность \\ \hline
		Нет & 0,633 \\ \hline
		Да & 0,379 \\ \hline
	\end{tabular}
  \end{center}
\end{table}

Практические результаты подтверждают теоретические выводы о том,
что нечеткая модель при применении $\Pi_f$ более эффективна по критерию
стабильности (\ref{trm:fuz-eff-extension-stab}).
