\section{Правило вывода}
Одной из целей исследования является определение формального правила вывода,
с помощью которого можно производить расчет прогнозной оценки
близости пользователя и объекта.

Современные РС оперирует с большим числом различной информации о
пользователях, которой может выступать не только информация об их
предпочтениях, как в случае с АКМ. Этой информацией может быть, к примеру,
социально-демографические данные пользователя, такая информация в существующих
исследованиях называется контекстной \cite{2d, citation}.
Разработанная модель позволяет работать
с подобной информацией, поэтому разработанная модель является расширением
АКМ по данным о пользователе. РС производит сопоставление
информации о пользователе и объекте, за счет которого определяется
оценка близости пользователя и объекта. В случае АКМ сопоставление
определяется эвристическими утверждениями. При работе с более широкой и
неоднозначной информацией о пользователе и при ее сопоставлении с данными об
объектах применяются знания эксперта, алгоритмические подходы такие, к примеру,
как машинное обучение.

В самом простом варианте сопоставление
является тождественным, и $X = Y$.
Такая ситуация возможна для РС, рекомендующей музыкальные произведения, где
характеристики пользователей и объектов --- это музыкальные жанры. Однако в общем случае
характеристики пользователей могут совершенно отличаться от характеристик
объектов. К примеру для банковской системы \cite{bank},
характеристиками пользователей являются социально-демографические данные,
а характеристики объектов --- финансовые данные.

В разработанной модели сопоставление информации и пользователе и объекте
определяется заданием
нечеткого отображения $\dc: X \times Y \rightarrow [0,1]$, именуемым оценкой
близости характеристик.

Функция сходства $\dc$ может быть задана в следующих формах:
\begin{enumerate}
\item табличной;
\item аналитической.
\end{enumerate}

Значения функции могут быть сформированы:
\begin{enumerate}
\item алгоритмически. К примеру, помощью методов классификации;
\item экспертно. С помощью экспертных знаний о прикладной области РС.
\end{enumerate}

С помощью функции $\dc$ можно определить нечеткое отображение $\Psi: X \rightarrow Y$ контента
пользователя на множество контентов объектов.
Для этого
зададим характеристическую функцию принадлежности $\nu_{\Psi}$ характеристики
$y$ нечеткому подмножеству универсального множества $Y$ при отображении $\Psi$ следующей формулой:
\begin{equation}
	\label{make-rho1}
		\nu_{\Psi}(y) = \underset{x \in X} {\mathrm{\max}} \min\{
			\delta_c(x,y); w_U(u, x) \}
\end{equation}

Определим функцию отображения контентов пользователей:
		\begin{equation}
			\Psi(c_X(u)) = \{ (y | \nu_{\Psi}(y)) \}, y \in Y
		\end{equation}

При наличии способа отображения контента пользователя, зададим расстояние
между пользователем $u$ и объектом $i$:
	\begin{equation}
		\label{my-rh}
		\rh(u,i) := 1 - \rhi(\overline{i}, i)\text{, где }
		c_Y(\overline{i}) := \Psi(c_X(u))
	\end{equation}
где $\overline{\rh}(u,i) $ --- расстояние вычисленное системой.


Будем говорить, что между пользователем $u$ и объектом $i$
выполняется отношение близости, если:
\begin{equation}
	\label{user-item-sim}
	u \R i\Leftrightarrow \text{ } \overline{\rho}(u,i) \ge
	\Delta_{\R}
\end{equation}

\begin{equation}
	\begin{aligned}
	\label{my-pi}
		\text{Нечеткое правило $\Pi_f$ нечеткой модели }\\
	\text{заключается в задании оценки сходства } \delta_c\\
	\text{и в дальнейшем расчете прогнозной функции }\rh
	\end{aligned}
\end{equation}

%\begin{trm}
%	При $\epsilon_0 \rightarrow 1$ отношение близости $\R$,
%	определенное формулой \ref{user-item-sim} в
%	контентной модели обладает свойством транзитивности.
%\end{trm}
%
%Расстояние между пользователем и объектом вычисляется через расстояние $\rhi$,
%то есть расстояние между пользователем $u$ и объектом $i$
%--- это расстояние между объектом $i$ и объектом $i^{\prime}$ таким, что
%$c_Y(i^{\prime}) = \Psi(c_X(u))$. Таким образом,
%если выполняется отношение близости  $i \rt i^{\prime}$, где
%$c_Y(i^{\prime}) = \Psi(c_X(u))$, то выполняется отношение
%близости $u \R i$. Выше было показано, что отношение близости $\rt$
%в контентной модели обладает свойством транзитивности\ref{trans-rho}. А потому отношение
%близости, заданное формулой \ref{user-item-sim} обладает свойством
%транзитивности.
