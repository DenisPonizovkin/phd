\section{Эффективное расширение двумерной РС с применением коллаборативных
техник}
Для того, чтобы использовать методы решения задач, которые используются в эвристических коллаборативных РС, необходимо
формализовать эти модели в нечеткой. Для этого представим контенты пользователей и объектов в виде нечетких множеств.
Представление контентов объектов будет являться аналогичным, как и для нечеткой модели. Контент пользователь во введенной модели запишется в виде
$u^k = \{ (i^1, \delta(a,i^1), ...,(i^{|I|}, \delta(a,i^{|I|}) ) \}$. 

\begin{assert}
Используя простые преобразования и заменяя функцию сходства на функцию расстояния, получим {\it нечеткие эвристичские коллаборативные РС}.
\end{assert}

\subsection{Повышение эффективности эвристических решений КРС за счет  использования метрических функций оценки сходства}
Как было показано в разделе <<Анализ коллаборативных РС>>, необходимым условием эффективности является выполнение 
транзитивности отношения сходства \ref{transAssert1, nec-cond-pred-srs, transAssert3}.
Это условие зависит от используемой функции в качестве оценки сходства. Во введенной модели в качестве такой оценки
используется функция расстояния, за счет чего выполняется отношение транзитивности отношения сходства\ref{trans-rho}. 
Поэтому использование алгоритмов эвристических коллаборативных РС при их формализации в нечеткой модели 
повышается эффективность получаемых решений, так как выполняются необходимые и  достаточные условия транзитивности. 

\subsection{Общий способ повышения эффективности решения задачи прогнозирования в СРС}
При создании реальной СРС по разным причинам условие сходства \ref{rho-sim} % или использование метрической функции для определения отношения сходства
может быть неприемлимым, так как накладывает слишком жесткое условие на разницу между значениями. Для такого, более общего случая 
(к примеру, для условия, что $\rho_1(u^1, u^2) < 1 - \Delta$), случая представим способ построения $\mathcal{N}^a$, при котором 
выполняется транзитивность отношения сходства между соседями, то есть выполняется необходимое и достаточное условие эффективности. 

Предварительно определим операцию пересечения и объединения 
контентов пользователей, для чего представим контент $u$ в виде
нечеткого множества:  $u = \{ i_0 : \delta_0(u, i_0) \}$, где $\delta_0(u, i_0) \ne \varnothing$.

\begin{equation}
u^1 \bigcup u^2 = \{ i_0 : \min(\delta_0(u^1, i_0), \delta_0(u^1, i_0)) \}
\end{equation}

\begin{equation}
u^1 \bigcap u^2 = \{ i_0 : \max(\delta_0(u^1, i_0), \delta_0(u^1, i_0)) \}
\end{equation}

После добавления нового элемента $u$ в множество проводится проверка условия:
\begin{equation}\label{method}
u \mathbf{R} a \wedge u \mathbf{R} u^{\bigcup} \wedge u \mathbf{R} u^{\bigcap}
\end{equation}
где $u^{\bigcup} = u^1 \bigcup u^2 \bigcup ... u^{M} \bigcup u, 
u^{\bigcap} = u^1 \bigcap u^2 \bigcap ... u^{M} \bigcap u, M = |\mathcal{N}^a|, u^m \in \mathcal{N}^a$
Если условие выполнено, то новый элемент остается в множестве, иначе он удаляется из множества.

\begin{assert}\label{method-assert}
В множестве соседей, построенном способом \ref{method} выполняется отношение транзитивности
\end{assert}

\begin{myproof}
Докажем при помощи метода индукции. 
\end{myproof}
Рассмотрим сначала ситуацию, когда множество объектов состоит из одного объекта $i$. По построению множества соседей  для нового пользователя $u$ 
выполняется $u \mathbf{R} u^{\bigcup} \wedge u \mathbf{R} u^{\bigcap}$, то есть
$\delta_0(u, i) - \mathrm{min}\{ \delta_0(u^m, i), u^m \in \mathcal{N}^a \} \le \varepsilon_0$ и
$\mathrm{max}\{ \delta_0(u^m, i), u^m \in \mathcal{N}^a\} - \delta_0(u, i) 
\le \varepsilon_0$ по определению расстояния и сходства пользователей \ref{eq:user-sim}.
% по определению расстояния. 

Покажем, что для нового элемента $u$, и $\forall$ $u^m \in \mathcal{N}^a$ верно
$|\delta_0(u, i) - \delta_0(u^m, i)| \le \varepsilon_0$. Для этого покажем, что
$|\delta_0(u, i) - \delta_0(u^m, i)| < |\delta_0(u, i) - \mathrm{min}\{ \delta_0(u^m, i), u^m \in \mathcal{N}^a \}|$:

\begin{enumerate}
\item Пусть $\delta_0(u, i) \ge \delta_0(u^m, i)$, тогда \\
$\delta_0(u, i) - \delta_0(u^m, i) < \delta_0(u, i) - \mathrm{min}\{ \delta_0(u^m, i), u^m \in \mathcal{N}^a \} \le \varepsilon_0$
\item Пусть $\delta_0(u, i) < \delta_0(u^m, i)$, тогда \\
$\delta_0(u^m, i) - \delta_0(u, i) < \mathrm{max}\{ \delta_0(u^m, i), u^m \in \mathcal{N}^a \} - 
\delta_0(u, i) \le \varepsilon_0$
\end{enumerate}

Пусть утверждение выполняется для $|I| = l-1$. Покажем, что для множества мощности $l$ выполняется
$\frac{1}{l} \cdot \sum \limits_{j=1}^{l} |\delta_0(u, i^j) - \delta_0(u^m, i^j)| \le \varepsilon_0$\\

Пусть $\delta_0(u, i) \ge \delta_0(u^m, i)$, тогда:\\

$\delta_0(u, u^m) = \sum \limits_{j=1}^{l-1} |\delta_0(u, i^j) - \delta_0(u^m, i^j)|$ + 
$(\delta_0(u, i) - \delta_0(u^m, i)) <$\\

$\sum \limits_{j=1}^{l-1} |\delta_0(u, i^j) - \delta_0(u^m, i^j)|$ 
+ $(\delta_0(u, i) - \mathrm{min}\{ \delta_0(u^m, i), u^m \in \mathcal{N}^a \}) <$  \\

$\sum \limits_{j=1}^{l-1} 
|\delta_0(u, i) - \mathrm{min}\{ \delta_0(u^m, i^j), u^m \in \mathcal{N}^a \}|$ 
+ $(\delta_0(u, i) - \mathrm{min}\{ \delta_0(u^m, i), u^m \in \mathcal{N}^a \}) = $  $\delta_0(u, u^{\bigcap}) \le \varepsilon_0$

При $\delta_0(u, i) < \delta_0(u^m, i)$ доказательство проводится 
аналогично приведенному с использованием максимума, а не минимума.

\begin{figure}[h]
\begin{algorithm}
\caption{Решение задачи прогнозирования при модифицированном алгоритме построения множества соседей прогнозируемого объекта $i_{\times}$}
\begin{algorithmic}[1]
\State $\mathcal{N}^at gets \varnothing$
\For {$i^l \in T$ $\wedge$ $\mu^t(i^l)$} \Comment{Соседи объекта ищутся среди тех, которые были оценены активным пользователем.
Для коллаборативных РС значением характеристической функции является оценка пользователя.}
  \If {$(\rho_1(i_{\times}, i^l) < 1 - \Delta$}\Comment{Объект является кандидатом на статус соседа.}
    \If {$|\mathcal{N}^at| = 0$}\Comment{Если множество соседей еще пусто, то добавляем кандидата $i^l$}
    \State $\mathcal{N}^at \gets \mathcal{N}^at \bigcup \{ i^l \}$
    \State $i^{\bigcup} \gets i^l \bigcup i_{\times}$
    \State $i^{\bigcap} \gets i^l \bigcap i_{\times}$
    \ElsIf{$i^l \mathbf{R} i_{\times} \wedge i^l \mathbf{R} i^{\bigcup} \wedge i^l \mathbf{R} i^{\bigcap}$}\Comment{Если 
множество соседей не пусто, то необходимо проверить условие \ref{trans-neigh}}
    \State $\mathcal{N}^at \gets \mathcal{N}^at \bigcup \{ i^k \}$
    \State $i^{\bigcup} \gets i^k \bigcup i^{\bigcup}$
    \State $i^{\bigcap} \gets i^k \bigcap i^{\bigcap}$
    \EndIf
  \EndIf
\EndFor
\State $\delta_{1}(a,i_{\times}) \gets p\{ \delta_0(i^k,i_{\times}) \}), i^k \in \mathcal{N}^a$
\end{algorithmic}
\end{algorithm}
\end{figure}
