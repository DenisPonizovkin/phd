
\section{Сравнение коллаборативной и нечеткой моделей рекомендательной системы}
Сравним эффективность моделей по критериям качества решения, вычислительной
сложности и стабильности.
%При реализации нечеткой модели разработчик
%единожды создает программный код самой модели. Если нужно что-то изменить,
%то меняется стратегия, которая заключается в задании функции $\delta_c$, но не код.
\subsection{Эффективность по критерию качества}
\begin{trm}
	\label{trm:fuz-eff-extension}
	Нечеткая модель является эффективным по критерию качества расширением АКМ.
\end{trm}
Нечеткая модель включает в себя АКМ по применяемым правилам вывода.
При этом коллаборативные правила вывода в нечеткой модели дают большую
эффективность, чем применение тех же правил в АКМ, так как в нечеткой
модели выполняются условия получения эффективного решения по критерию качества
при применении коллаборативных правил вывода, что было показано в разделе
<<Применение коллаборативных правил вывода в нечеткой модели>>.

Эффективность нечеткой модели по критерию качества при применении правила
вывода $\Pi_f$ так же, как и в коллаборативных моделях ограничена
дополнительным условием. Для нечеткой модели и $\Pi_f$ --- это требования
к заданию $\Pi_f$, которые выражены достаточными условиями
(\ref{fuz-cond-pred}) и (\ref{fuz-cond-topn}). Как и в АКМ выполнение этого условия
зависит от разработчиков системы. Однако в АКМ разработчик
обладает меньше свободой действий, чем в нечеткой модели. Все, что доступно
разработчику в АКМ --- это выбор функции, которая будет
использоваться в качестве меры сходства, и ее порогового значения. В нечеткой
модели разработчик выбирает алгоритмы, эвристики, наборы данных, знания и т.п.,
чтобы задать правила вывода. При этом было показано, что если правило вывода
$\Pi_f$ задано так, что выполняются (\ref{fuz-cond-pred}) и
(\ref{fuz-cond-topn}), то решение будет гарантированно эффективно.
Для АКМ такую гарантию дать нельзя, так как существует зависимость от исходных
данных. Поэтому теоретически нечеткая модель более
эффективна, чем АКМ, и является ее расширением, так как включает
в себя коллаборативные правила вывода, и может применяться на любых данных
о пользователе $X$ и объекте, а не только тога, когда $X = I$ И $Y = X$.

Правила вывода РС задают взаимосвязь между информацией о пользователе и об
объекте. В правилах вывода
АКМ эта взаимосвязь заложена статически в качестве эвристик, выполнение
которых не гарантировано. Для нечеткой модели взаимосвязь является
динамической. Ее можно рассматривать как паттерн программирования
<<стратегия>>. При реализации нечеткой модели разработчик
единожды создает программный код самой модели. Если нужно что-то изменить,
то меняется стратегия, но не код.

\subsection{Эффективность по критерию стабильности}
\begin{trm}
	\label{trm:fuz-eff-extension-stab}
	Нечеткая модель является эффективным расширением АКМ
	по критерию стабильности.
\end{trm}

Эффективность по качеству решений, получаемых в АКМ зависит от свойств исходных
данных, поэтому АКМ неэффективны по критерию стабильности. В том числе, если
мощность $P$ мала или вовсе равна нулю, то АКМ даже нельзя применять для
решения.

При применении
правила вывода $\Pi_f$ в нечеткой модели эффективность по качеству решения
не зависит от свойств исходных данных, зависит только от разработчика.

\subsection{Эффективность по критерию вычислительной сложности}
  \begin{table}[h]
	  \begin{center}
		  \caption{Асимптотические сложности алгоритмов решений}
		  \label{ass-comp}
		  \begin{tabular}{|c|c|c|c|}
			  \hline
			  Модель   & Правило вывода & Задача & Сложность  \\ \hline
			  ООМ      & $\Pi_{O}$ & $topN$ & $O(n^2)$  \\ \hline
			  СОМ      & $\Pi_{C}$ &Прогнозирование & $O(m)$ \\ \hline
			  Нечеткая & $\Pi_{O}$ &$topN$ & $O(n^2)$  \\ \hline
			  Нечеткая & $\Pi_{C}$ &Прогнозирование & $O(m)$ \\ \hline
			  Нечеткая & $\Pi_{f}$ &$topN$ & $O(n)$  \\ \hline
			  Нечеткая & $\Pi_{f}$ &Прогнозирование & $O(C)$ \\ \hline
		  \end{tabular}
	  \end{center}
  \end{table}

\begin{assert}
	\label{ass:eff-calc}
	Нечеткая модель является эффективным расширением АКМ
	по критерию вычислительной сложности.
\end{assert}

\subsection{Выводы}
В ходе диссертационного исследования была разработана
математическая модель РС, которая является эффективным
расширением АКМ по всем заданным критериям.
%\begin{enumerate}
%	\item Зависимость от свойств исходных данных: динамики, неоднородности,
%		мощности. Решения в нечеткой контентной модели в общем случае не
%		зависят от свойств исходных данных. Если мощность исходных данных мала,
%		то АКМ не применимы для решения задач. Если исходные данные обладают
%		свойством динамики или неоднородности, то АКМ не гарантируют получения
%		качественного решения.
%
%	\item Асимптотическая сложность. Алгоритмы решений задач нечеткой контентной модели
%		обладают асимптотической сложность, меньшей на порядок асимптотических
%		сложностей
%		алгоритмов решений АКМ;
%
%	\item Качество решений. Качество решений в контентной нечеткой модели
%		зависит от разработчика --- от аккуратности задания $\delta_c$.
%		Если функция задана аккуратно, то решения качественны.
%
%		Качество решений в АКМ зависят от разработчика --- от выбора
%		$\du, \di$ и $\Delta_u, \Delta_i$. Если эти параметры выбраны так, что
%		выполняются достаточные условия, то АКМ не гарантирует получение
%		качественных решений, так как это зависит в том числе и от свойств
%		исходных данных.
%\end{enumerate}
