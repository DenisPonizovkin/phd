%c
% ОЦЕНКА ЭФФЕКТИВНОСТИ
%
\section{Обобщенная оценка эффективности по критерию качества}
В разделе <<Анализ коллаборативной фильтрации и оценок>> было
показано,
что не всегда существующие оценки $\eit$
могут быть применены для выявление точности решения задачи $topN$
(\ref{eit-use}). В том числе существует открытая проблема отсутствия стандартной
оценки (\ref{no-standard-eval}).

В данном разделе предложена оценка,
основанная на правиле вывода $\Pi_f$, то есть для того, чтобы
определить качество решения вычисляется расстояние между активным
пользователем и объектами результирующего множества.

%Пусть задано такое расстояние $\overline{\rho}$.
%которое обладает слеюующим
%свойством:
%\begin{equation}\label{rho-pro}
%|\overline{\rho}(u_a, i) - \rho(u_a, i)| \le \varepsilon_p \in \varepsilon(0)
%\end{equation}

Определим оценку следующим образом:
\begin{equation}
	\label{content-eff}
	\ec = \frac{1}{|\overline{P}^a_{\bot}|} \cdot
	\sum \limits_{i_{\bot} \in \overline{P}^a_{\bot}}
	|\rh(u_a, i) - \rho(u_a, i_{\bot})|,
	(i , \rh(u_a, i)) \in \overline{P}^a_{\bot},
	(i_{\bot}, \rho(u_a, i_{\bot})) \in P^a_{\bot},
\end{equation}
где множества $\overline{P}^a_{\bot}, P^a_{\bot}$ упорядочены по возрастанию
значений $\rh,\rho$ соответственно для оценки задачи $topN$.
Для оценки задачи прогнозирования сортировка множеств не проводится, так
как $i = i_{\bot}$.

\subsection{Свойства оценки}
Покажем в данном разделе, что значения введенной
оценки
могут служить объективным показателем качества решения
решения задач $topN$ и
прогнозирования, а также, что введенная оценка
коррелируют с используемыми оценками
$\eap$, $\eit$ и $\eat$.

\subsubsection{Метрические свойства}
\begin{trm}
Оценка $\ec$ обладает метрическими свойствами.
\end{trm}
Если рассматривать решение как прототип контента пользователя,
а тестовое и результирующие множества как нечеткие подмножества
множества объектов, где характеристическая функция --- это
$\rho$ или $\rh$ соответственно для тестового и результирующего
множеств, то оценка $\ec$ будет обобщенным расстоянием Хэмминга
между двумя нечеткими подмножествами множества объектов. Выше было показано,
что такое расстояние обладает метрическими свойствами (\ref{metric}).
Так как функция $\ec$ обладает метрическими свойствами, то ее значение
является количественным показателем качества решения \cite{fdca}.

\subsubsection{Объективность показателя качества решения задачи прогнозирования}
Покажем, что оценка $\ec$ может применяться для оценки задачи
прогнозирования.
\begin{trm}
Значение оценки $\ec$ является объективным показателем аккуратности решения
	задачи прогнозирования.
\end{trm}

Объективность оценки $\ec$ следует из того, что $\ec$ и $\eap$
эквивалентны.
Выше было сказано, что значения оценки $\eap$ соответствуют задачи
прогнозирования, а потому являются объективным показателем аккуратности
решения. Выше было показано, что оценка $\ec$ является количественным
показателем аккуратности. А, так как $\ec$ и $\eap$ эквивалентны,
то оценка $\ec$ может применяться для оценки задачи прогнозирования.
Покажем, что корреляция существует.

Для оценки задачи
прогнозирования $i_{\bot} = i$, поэтому
$\ec
= MAE = \eap$.

\subsubsection{Объективность показателя качества решения задачи $topN$}
Оценка $\ec$ определяет
погрешность значений составленной функции $\rh$, что удовлетворяет задаче
прогнозирования, однако для оценки задачи $topN$ нужно понимать, какое число
рекомендованных объектов близко к пользователю. Рассмотрим возможность
применения оценки $\ec$ для определения точности к задаче $topN$.
$P^a_{\bot} = \{(i_{\bot}| \rho(u_a,i_{\bot})): \rho(u_a,  i_{\bot}) \le
\varepsilon_{\R}$,
поэтому мы можем определить,
насколько точно задана функция $\rh$ по отношению к объектам, между которыми и
активным пользователем выполняется отношение близости.

%Показателем этой
%точности будем называть среднюю погрешность $\midel = \frac{1}{|\test|} \sum
%\limits_{(i_{\bot}|\rho(u_a,i_{\bot}) \in \test} |\rh(u_a,i_{\bot}) -
%\rho(u_a,i_{\bot})|$.

%\begin{defn}
%	Функция $\rh$ задана с точностью $\midel$ в рамках решения задачи $topN$,
%	что означает, что $|\rh(u_a,i) - \rho(u_a,i)| \le \midel$.
%\end{defn}

\begin{trm}
Если $\Pi_f$ задано аккуратно и
если решение эффективно по $\ec$, то решение эффективно и по $\eat$.
\end{trm}

Рассмотрим случаи идеального решения (когда решение эффективно для
$\varepsilon_{topN}$) и идеального задания $\Pi_f$ (когда $\varepsilon_p = 0$).
Тогда получим, что
$(\rh(u_a, i) = \rho(u_a, i) \wedge (\rh(u_a, i) = \rho(u_a, i_{\bot})$.
По исходным данным задачи известно, что $\rho(u_a, i_{\bot}) \ge \Delta_{\R}$.
Поэтому $\rho(u_a, i) \ge \Delta_{\R}$, а потому решение эффективно и по $\eat$.

Пусть решение эффективно по $\eat$, тогда $\rho(u_a, i) \ge \Delta_{\R}$.
При идеальном задании правила вывода (когда $\varepsilon_p = 0$), $\Delta_{\R} = 1$ получим,
что $\rh(u_a, i) = \rho(u_a, i)) \ge \Delta_{\R}$, а по тому
$\ec \le 1 - \Delta_{\R}$, а пороговым значением $\ec$ является величины
$1 - \Delta_{\R}$. В частном случае, когда $\Delta_{\R} = 1$, $\ec \le 0$,
то есть решение идеально.

\subsection{Соотношение с объектно-ориентированной оценкой точности решения задачи $topN$}
Покажем, что в тех случаях, когда для оценки точности решения задачи $topN$
может применяться $\eit$ --- то есть исходные данные таковы,
что выполняется утверждение (\ref{eit-use}), --- верно,
что если $\eit \le \varepsilon_{topN}$, то решение точно по $\ec$,
и, если решение точно по  $\ec$, то $\eit \le \varepsilon_{topN}$.

Рассмотрим идеальную ситуацию, когда $\Delta_{\R} = 1, \varepsilon_p =
\varepsilon_{topN} = 0$. Так как решение эффективно по $\ec$, то
$\rho(u_a, i) = \rh(u_a, i) = \rho(u_a, i_{\bot}) = 1$. Функции $\rh, \rhi$
являются метрическими функциями, а $\rh$ выражается через $\rhi$:
$\rh(u,i) = 1 - \overset{i}{\mathcal{\rhi}}(\overline{i}, i)\text{, где }
c_Y(\overline{i}) = \Psi(c_X(u))$. Так как $\rho(u_a, i) = \rh(u_a, i) = \rho(u_a, i_{\bot}) = 1$
и $\rhi$ обладает свойством неравенства треугольника, то $\rhi(i, i_{\bot}) \le
0$, то есть $i \rt i_{\bot}$, и поэтому решение эффективно по $\eit$.

Так как решение эффективно по $\eit$, то
$\exists (i_{\bot} (i_{\bot}, \rho(u_a, i_{\bot})) \in \test: i \rt i_{\bot}$,
а так как выполняется утверждение (\ref{eit-use}), то $\rho(u_a, i) \ge
\Delta_{\R}$. По виду исходных данных задачи $topN$ выполняется, что
$\rho(u_a, i_{\bot}) \ge \Delta_{\R}$. В рассматриваемой идеальной ситуации
(когда $\Delta_{\R} = 1$) получим, что $\ec = 0$, то есть решение эффективно и по $\ec$.

%Пусть решение эффективно по $\ec$.
%В крайнем случае, когда $\varepsilon_p = 0$, $\Delta_{\R} = 1$ получим,
%что $\rh(u_a, i) = \rho(u_a, i)) = \rho(u_a, i_{\bot}) = 1$. В силу метрических
%свойств расстояния получим, что $\rho(i, i_{\bot}) \le \rho(u_a, i)
%+ \rho(u_a, i_{\bot}) = 2$.

\section{Определение нечеткой модели}
\begin{defn}
	\label{content-model}
	Контентная модель --- это $(c_X, c_Y, \Pi, \ec)$,
\end{defn}
где $\Pi \in \{\Pi_f, \Pi_O, \Pi_C
	\}$ --- правила вывода контентной нечеткой модели включают
	в себя коллаборативные правила вывода, но при использовании
	обобщенных расстояний Хэмминга (\ref{rhu-item}), (\ref{rhu-user}),
	а также правило вывода (\ref{content-solve-tech}), основанное
	на определении функции $\delta_c$. Контенты являются нечеткими подмножествами
	множеств характеристик. При использовании коллаборативных правил вывода
	универсальным множеством характеристик пользователей является множество
	объектов.

	Из описания нечеткой модели видно, что нечеткая модель включает в себя КРС.

%\subsection{Результат}
%Разработана нечеткая контентная модель РС, которая является расширением
%КРС, так как влюкчает в себя правила вывода КРС, а правило вывода
%(\ref{content-solve-tech}) не зависит от мощности исходного множества.

%\begin{enumerate}
%	\item Разработаны модификации алгоритмов КФ
%		при использовании
%		представления данных методами контентной модели так, что
%		гарантируется выполнение необходимых и достаточных
%		условий точности или аккуратности решений
%\label{transAssert1},
%\label{ass:suf-topnsolve-ors},
%\label{suff-cond-pred-ors},
%\label{nec-cond-pred-srs},
%\label{suf-cond-pred-srs}.
%
%\end{enumerate}

%% Будем говорить, что множество входных данных {\it репрезентативно}, если 
%% $\underset{k \rightarrow \infty} {\mathrm{\lim}}$  $|\overline{\delta} - \delta_0(u_a,i^k)| = 0$ и 
%% $\underset{k \rightarrow \infty} {\mathrm{\lim}}$  $|\overline{\rho} - \rho(u_a,i^k)| = 0$, где
%% $\overline{\delta}, \overline{\rho}$ --- средняя величина оценки сходства и расстояния соответственно, $i^k \in I, k=1..|I|$.
%% Будем считать, что входные и тестовые множества репрезентативны.



