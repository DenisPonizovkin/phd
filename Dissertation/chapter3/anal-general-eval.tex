\section{Обобщенные оценки эффективности}
Прежде, чем приступить к анализу оценок эффективности, обобщим их. Это возможно сделать, так
как функции, принадлежащие одной группе коррелируют друг с другом \cite{herloker-eval}.

\subsection{Обобщение оценок эффективности задачи прогнозирования}
Функции данной группы зависят от разницы между спрогнозированной оценкой и настоящей.
Введем обобщенную оценку эффективности для этой группы как $\mathcal{E}_p(\delta_0(a, i), \delta_1(a, i))$.

\subsection{Обобщение оценок эффективности задачи $\top$}
Определим оценку, принадлежащую классу оценок эффективности задачи $\top$ символом $\mathcal{E}_{\top}$.
Будем говорить, что решение эффективно, если
\begin{equation}
\mathcal{E}_{\top} \le \epsilon_1
\end{equation}
где $\epsilon_1$ --- некоторая заданная малая величина.

Значение приведенных оценок эффективности зависит от числа $K = |\{ i_1 \in T^a_1: a \mathit{R} i \}|$. 
Чем больше число $K$, тем значение $\mathcal{E}_{\top}$ меньше и тем решение эффективней.

\begin{assert}
Решение эффективно, если $T^a_{\times}$ такое, что выполняется $a \mathit{R} i^n$, где $n=1..K$, $K \ge N \cdot (1 - \epsilon_1)$
\end{assert}

Покажем, что утверждение верно для каждой приведенной оценки эффективности:
\begin{itemize}
\item Точность: $\mathcal{E}_{\top} \le 1 - \frac{K}{N} = 1 - (1 - \epsilon_1) = \epsilon_1$

\item Полнота: так как рассматривается при $N = M$, то доказательство сводится к доказательству, относящемуся к  точности.

\item Точность $P@L$ --- верно, так как данная оценка является частным случаем точности. Точность может рассматриваться как $P@N$.
Утверждение верно для $P@N$, то утверждение верно и для $P@L$, так как $L \le N$.

\item $AveP$: так как отношение сходства выполняется для $i^n, n=1..K$, то $P@L = 0, l=1..K$ и 
$\mathcal{E}_{\top} = \frac{1}{K} \cdot \sum \limits_{L=K}^{N} P@L$. В худшем случае, когда $P@L=1, l=(K+1)..N$ получим, что
$\mathcal{E}_{\top} = \frac{N-K}{K} = \frac{N}{K} - 1 \le \frac{N}{N \cdot (1 - \epsilon_1)} - 1 = \frac{1}{1 - \epsilon_1} - 1$.
В предельном моменте, при $\epsilon_1 \rightarrow 1$ решение эффективно для люых значений $\epsilon_1$.

\item $IDCG = 1 + \sum \limits_{n=2}^N \frac{1}{log_2(n)}$. Так как $s(n)=1, n=1..K$, то $DCG=1 + \sum \limits_{n=2}^K \frac{1}{log_2(n)}$.
$NDCG = 1 - \frac{1 + \sum \limits_{n=2}^K \frac{1}{log_2(n)}}{1 + \sum \limits_{n=2}^N \frac{1}{log_2(n)}}$. В предельном случае, при $\epsilon_1 \rightarrow 0$ получим, что $NDCG=0$, то есть решение эффективно.
\end{itemize}

Таким образом, будем рассматривать оценки эффективности задачи $\top$, в общем, в виде как функцию 
$\mathcal{E}^a_{\top}(K)$ и $\mathcal{E}^i_{\top}(K)$, в зависимости от задания $s(n)$. В первом случае речь идет о 
целевой обобщенной функции, во втором --- об объектно-ориентированной.
