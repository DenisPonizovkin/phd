\chapter*{Заключение}						% Заголовок
\addcontentsline{toc}{chapter}{Заключение}	% Добавляем его в оглавление

%% Согласно ГОСТ Р 7.0.11-2011:
%% 5.3.3 В заключении диссертации излагают итоги выполненного исследования, рекомендации, перспективы дальнейшей разработки темы.
%% 9.2.3 В заключении автореферата диссертации излагают итоги данного исследования, рекомендации и перспективы дальнейшей разработки темы.
%% Поэтому имеет смысл сделать эту часть общей и загрузить из одного файла в автореферат и в диссертацию:
В рамках проведенного диссертационного исследования решена актуальная
научно-техническая задача построения математической модели рекомендательной
системы. Теоретически и практически показано, что разработанная модель
является эффективным расширением существующей
анамнестической коллаборативной модели.

В ходе диссертационного исследования были решены все поставленные
цели и задачи:

\begin{enumerate}
	\item
		проведен анализ эффективности АКМ по критериям оценки
		эффективности моделей РС. Показано, что в общем случае
		анамнестическая коллаборативная модель не является
		эффективной моделью рекомендательной системы;
	\item разработана модель рекомендательной системы на нечетких
		множествах и алгоритмы решений задач в рамках этой модели.
		Показано, что разработанная модель является эффективным расширением
		анамнестической коллаборативной модели;
	\item разработано программное обеспечение, с помощью которого
		было проведено тестирование разработанной и анамнестической
		коллаборативной моделей. Результаты тестирования подтверждают
		теоретический вывод о том, что разработанная модель более эффективна,
		чем анамнестическая коллаборативная модель.
\end{enumerate}
